%%
%% Automatically generated file from DocOnce source
%% (https://github.com/hplgit/doconce/)
%%
% #ifdef PTEX2TEX_EXPLANATION
%%
%% The file follows the ptex2tex extended LaTeX format, see
%% ptex2tex: http://code.google.com/p/ptex2tex/
%%
%% Run
%%      ptex2tex myfile
%% or
%%      doconce ptex2tex myfile
%%
%% to turn myfile.p.tex into an ordinary LaTeX file myfile.tex.
%% (The ptex2tex program: http://code.google.com/p/ptex2tex)
%% Many preprocess options can be added to ptex2tex or doconce ptex2tex
%%
%%      ptex2tex -DMINTED myfile
%%      doconce ptex2tex myfile envir=minted
%%
%% ptex2tex will typeset code environments according to a global or local
%% .ptex2tex.cfg configure file. doconce ptex2tex will typeset code
%% according to options on the command line (just type doconce ptex2tex to
%% see examples). If doconce ptex2tex has envir=minted, it enables the
%% minted style without needing -DMINTED.
% #endif

% #define PREAMBLE

% #ifdef PREAMBLE
%-------------------- begin preamble ----------------------

\documentclass[%
twoside,                 % oneside: electronic viewing, twoside: printing
final,                   % or draft (marks overfull hboxes, figures with paths)
10pt]{article}

\listfiles               % print all files needed to compile this document

\usepackage{relsize,makeidx,color,setspace,amsmath,amsfonts}
\usepackage[table]{xcolor}
\usepackage{bm,microtype}

\usepackage{ptex2tex}

% #ifdef MINTED
\usepackage{minted}
\usemintedstyle{trac}
% #endif

\usepackage[T1]{fontenc}
%\usepackage[latin1]{inputenc}
\usepackage[utf8]{inputenc}

\usepackage{lmodern}         % Latin Modern fonts derived from Computer Modern

% Hyperlinks in PDF:
\definecolor{linkcolor}{rgb}{0,0,0.4}
\usepackage[%
    colorlinks=true,
    linkcolor=linkcolor,
    urlcolor=linkcolor,
    citecolor=black,
    filecolor=black,
    %filecolor=blue,
    pdfmenubar=true,
    pdftoolbar=true,
    bookmarksdepth=3   % Uncomment (and tweak) for PDF bookmarks with more levels than the TOC
            ]{hyperref}
%\hyperbaseurl{}   % hyperlinks are relative to this root

\setcounter{tocdepth}{2}  % number chapter, section, subsection

\usepackage[framemethod=TikZ]{mdframed}

% --- begin definitions of admonition environments ---

% --- end of definitions of admonition environments ---

% prevent orhpans and widows
\clubpenalty = 10000
\widowpenalty = 10000

% --- end of standard preamble for documents ---


% insert custom LaTeX commands...

\raggedbottom
\makeindex

%-------------------- end preamble ----------------------

\begin{document}

% #endif


% ------------------- main content ----------------------



% ----------------- title -------------------------

\thispagestyle{empty}

\begin{center}
{\LARGE\bf
\begin{spacing}{1.25}
Slides from FYS-KJM4411/9411 Variational Monte Carlo methods
\end{spacing}
}
\end{center}

% ----------------- author(s) -------------------------

\begin{center}
{\bf Morten Hjorth-Jensen${}^{1, 2}$} \\ [0mm]
\end{center}

    \begin{center}
% List of all institutions:
\centerline{{\small ${}^1$Department of Physics, University of Oslo, Oslo, Norway}}
\centerline{{\small ${}^2$Department of Physics and Astronomy and National Superconducting Cyclotron Laboratory, Michigan State University, East Lansing, MI 48824, USA}}
\end{center}
    
% ----------------- end author(s) -------------------------

\begin{center} % date
Spring 2015
\end{center}

\vspace{1cm}


% !split
\subsection{Quantum Monte Carlo Motivation}

% --- begin paragraph admon ---
\paragraph{}
Given a hamiltonian $H$ and a trial wave function $\Psi_T$, the variational principle states that the expectation value of $\langle H \rangle$, defined through 
\[
   E[H]= \langle H \rangle =
   \frac{\int d\bm{R}\Psi^{\ast}_T(\bm{R})H(\bm{R})\Psi_T(\bm{R})}
        {\int d\bm{R}\Psi^{\ast}_T(\bm{R})\Psi_T(\bm{R})},
\]
is an upper bound to the ground state energy $E_0$ of the hamiltonian $H$, that is 
\[
    E_0 \le \langle H \rangle .
\]
In general, the integrals involved in the calculation of various  expectation values  are multi-dimensional ones. Traditional integration methods such as the Gauss-Legendre will not be adequate for say the  computation of the energy of a many-body system.
% --- end paragraph admon ---



% !split
\subsection{Quantum Monte Carlo Motivation}

% --- begin paragraph admon ---
\paragraph{}
The trial wave function can be expanded in the eigenstates of the hamiltonian since they form a complete set, viz.,
\[
   \Psi_T(\bm{R})=\sum_i a_i\Psi_i(\bm{R}),
\]
and assuming the set of eigenfunctions to be normalized one obtains 
\[
     \frac{\sum_{nm}a^*_ma_n \int d\bm{R}\Psi^{\ast}_m(\bm{R})H(\bm{R})\Psi_n(\bm{R})}
        {\sum_{nm}a^*_ma_n \int d\bm{R}\Psi^{\ast}_m(\bm{R})\Psi_n(\bm{R})} =\frac{\sum_{n}a^2_n E_n}
        {\sum_{n}a^2_n} \ge E_0,
\]
where we used that $H(\bm{R})\Psi_n(\bm{R})=E_n\Psi_n(\bm{R})$.
In general, the integrals involved in the calculation of various  expectation
values  are multi-dimensional ones. 
The variational principle yields the lowest state of a given symmetry.
% --- end paragraph admon ---




% !split
\subsection{Quantum Monte Carlo Motivation}

% --- begin paragraph admon ---
\paragraph{}
In most cases, a wave function has only small values in large parts of 
configuration space, and a straightforward procedure which uses
homogenously distributed random points in configuration space 
will most likely lead to poor results. This may suggest that some kind
of importance sampling combined with e.g., the Metropolis algorithm 
may be  a more efficient way of obtaining the ground state energy.
The hope is then that those regions of configurations space where
the wave function assumes appreciable values are sampled more 
efficiently.
% --- end paragraph admon ---




% !split
\subsection{Quantum Monte Carlo Motivation}

% --- begin paragraph admon ---
\paragraph{}
The tedious part in a VMC calculation is the search for the variational
minimum. A good knowledge of the system is required in order to carry out
reasonable VMC calculations. This is not always the case, 
and often VMC calculations 
serve rather as the starting
point for so-called diffusion Monte Carlo calculations (DMC). DMC is a way of
solving exactly the many-body Schroedinger equation by means of 
a stochastic procedure. A good guess on the binding energy
and its wave function is however necessary. 
A carefully performed VMC calculation can aid in this context.
% --- end paragraph admon ---




% !split
\subsection{Quantum Monte Carlo Motivation}

% --- begin paragraph admon ---
\paragraph{}
\begin{itemize}
\item Construct first a trial wave function $\psi_T(\bm{R},\bm{\alpha})$,  for a many-body system consisting of $N$ particles located at positions 
\end{itemize}

\noindent
$\bm{R}=(\bm{R}_1,\dots ,\bm{R}_N)$. The trial wave function depends on $\alpha$ variational parameters $\bm{\alpha}=(\alpha_1,\dots ,\alpha_M)$.
\begin{itemize}
\item Then we evaluate the expectation value of the hamiltonian $H$ 
\end{itemize}

\noindent
\[
   E[H]=\langle H \rangle =
   \frac{\int d\bm{R}\Psi^{\ast}_{T}(\bm{R},\bm{\alpha})H(\bm{R})\Psi_{T}(\bm{R},\bm{\alpha})}
        {\int d\bm{R}\Psi^{\ast}_{T}(\bm{R},\bm{\alpha})\Psi_{T}(\bm{R},\bm{\alpha})}.
\]
\begin{itemize}
\item Thereafter we vary $\alpha$ according to some minimization algorithm and return to the first step.
\end{itemize}

\noindent
% --- end paragraph admon ---




% !split
\subsection{Quantum Monte Carlo Motivation}

% --- begin paragraph admon ---
\paragraph{Basic steps.}
Choose a trial wave function
$\psi_T(\bm{R})$.
\[
   P(\bm{R})= \frac{\left|\psi_T(\bm{R})\right|^2}{\int \left|\psi_T(\bm{R})\right|^2d\bm{R}}.
\]
This is our new probability distribution function  (PDF).
The approximation to the expectation value of the Hamiltonian is now 
\[
   E[H(\bm{\alpha})] = 
   \frac{\int d\bm{R}\Psi^{\ast}_T(\bm{R},\bm{\alpha})H(\bm{R})\Psi_T(\bm{R},\bm{\alpha})}
        {\int d\bm{R}\Psi^{\ast}_T(\bm{R},\bm{\alpha})\Psi_T(\bm{R},\bm{\alpha})}.
\]
% --- end paragraph admon ---




% !split
\subsection{Quantum Monte Carlo Motivation}

% --- begin paragraph admon ---
\paragraph{}
Define a new quantity
\[
   E_L(\bm{R},\bm{\alpha})=\frac{1}{\psi_T(\bm{R},\bm{\alpha})}H\psi_T(\bm{R},\bm{\alpha}),
   \label{eq:locale1}
\]
called the local energy, which, together with our trial PDF yields
\[
  E[H(\bm{\alpha})]= = \int P(\bm{R})E_L(\bm{R}) d\bm{R}\approx \frac{1}{N}\sum_{i=1}^NP(\bm{R_i},\bm{\alpha})E_L(\bm{R_i},\bm{\alpha})
  \label{eq:vmc1}
\]
with $N$ being the number of Monte Carlo samples.
% --- end paragraph admon ---








% !split
\subsection{Quantum Monte Carlo}

% --- begin paragraph admon ---
\paragraph{}
The Algorithm for performing a variational Monte Carlo calculations runs thus as this

\begin{itemize}
   \item Initialisation: Fix the number of Monte Carlo steps. Choose an initial $\bm{R}$ and variational parameters $\alpha$ and calculate $\left|\psi_T^{\alpha}(\bm{R})\right|^2$. 

   \item Initialise the energy and the variance and start the Monte Carlo calculation.
\begin{itemize}

      \item Calculate  a trial position  $\bm{R}_p=\bm{R}+r*step$ where $r$ is a random variable $r \in [0,1]$.

      \item Metropolis algorithm to accept or reject this move  $w = P(\bm{R}_p)/P(\bm{R})$.

      \item If the step is accepted, then we set $\bm{R}=\bm{R}_p$. 

      \item Update averages

\end{itemize}

\noindent
   \item Finish and compute final averages.
\end{itemize}

\noindent
Observe that the jumping in space is governed by the variable \emph{step}. This is Called brute-force sampling.
Need importance sampling to get more relevant sampling, see lectures below.
% --- end paragraph admon ---



% !split
\subsection{Quantum Monte Carlo: hydrogen atom}

% --- begin paragraph admon ---
\paragraph{}
The radial Schroedinger equation for the hydrogen atom can be
written as
\[
-\frac{\hbar^2}{2m}\frac{\partial^2 u(r)}{\partial r^2}-
\left(\frac{ke^2}{r}-\frac{\hbar^2l(l+1)}{2mr^2}\right)u(r)=Eu(r),
\]
or with dimensionless variables
\[
-\frac{1}{2}\frac{\partial^2 u(\rho)}{\partial \rho^2}-
\frac{u(\rho)}{\rho}+\frac{l(l+1)}{2\rho^2}u(\rho)-\lambda u(\rho)=0,
\label{eq:hydrodimless1}
\]
with the hamiltonian
\[
H=-\frac{1}{2}\frac{\partial^2 }{\partial \rho^2}-
\frac{1}{\rho}+\frac{l(l+1)}{2\rho^2}.
\]
Use variational parameter $\alpha$ in the trial
wave function 
\[
   u_T^{\alpha}(\rho)=\alpha\rho e^{-\alpha\rho}. 
   \label{eq:trialhydrogen}
\]
% --- end paragraph admon ---



% !split
\subsection{Quantum Monte Carlo: hydrogen atom}

% --- begin paragraph admon ---
\paragraph{}
Inserting this wave function into the expression for the
local energy $E_L$ gives
\[
   E_L(\rho)=-\frac{1}{\rho}-
              \frac{\alpha}{2}\left(\alpha-\frac{2}{\rho}\right).
\]
A simple variational Monte Carlo calculation results in

\begin{quote}
\begin{tabular}{cccc}
\hline
\multicolumn{1}{c}{ $\alpha$ } & \multicolumn{1}{c}{ $\langle H \rangle $ } & \multicolumn{1}{c}{ $\sigma^2$ } & \multicolumn{1}{c}{ $\sigma/\sqrt{N}$ } \\
\hline
7.00000E-01 & -4.57759E-01         & 4.51201E-02 & 6.71715E-04       \\
8.00000E-01 & -4.81461E-01         & 3.05736E-02 & 5.52934E-04       \\
9.00000E-01 & -4.95899E-01         & 8.20497E-03 & 2.86443E-04       \\
1.00000E-00 & -5.00000E-01         & 0.00000E+00 & 0.00000E+00       \\
1.10000E+00 & -4.93738E-01         & 1.16989E-02 & 3.42036E-04       \\
1.20000E+00 & -4.75563E-01         & 8.85899E-02 & 9.41222E-04       \\
1.30000E+00 & -4.54341E-01         & 1.45171E-01 & 1.20487E-03       \\
\hline
\end{tabular}
\end{quote}

\noindent
% --- end paragraph admon ---




% !split
\subsection{Quantum Monte Carlo: hydrogen atom}

% --- begin paragraph admon ---
\paragraph{}

We note that at $\alpha=1$ we obtain the exact
result, and the variance is zero, as it should. The reason is that 
we then have the exact wave function, and the action of the hamiltionan
on the wave function
\[
   H\psi = \mathrm{constant}\times \psi,
\]
yields just a constant. The integral which defines various 
expectation values involving moments of the hamiltonian becomes then
\[
   \langle H^n \rangle =
   \frac{\int d\bm{R}\Psi^{\ast}_T(\bm{R})H^n(\bm{R})\Psi_T(\bm{R})}
        {\int d\bm{R}\Psi^{\ast}_T(\bm{R})\Psi_T(\bm{R})}=
\mathrm{constant}\times\frac{\int d\bm{R}\Psi^{\ast}_T(\bm{R})\Psi_T(\bm{R})}
        {\int d\bm{R}\Psi^{\ast}_T(\bm{R})\Psi_T(\bm{R})}=\mathrm{constant}.
\]
\textbf{This gives an important information: the exact wave function leads to zero variance!}
Variation is then performed by minimizing both the energy and the variance.
% --- end paragraph admon ---





% !split
\subsection{Quantum Monte Carlo: the helium atom}

% --- begin paragraph admon ---
\paragraph{}
The helium atom consists of two electrons and a nucleus with
charge $Z=2$. 
The contribution  
to the potential energy due to the attraction from the nucleus is
\[
   -\frac{2ke^2}{r_1}-\frac{2ke^2}{r_2},
\] 
and if we add the repulsion arising from the two 
interacting electrons, we obtain the potential energy
\[
 V(r_1, r_2)=-\frac{2ke^2}{r_1}-\frac{2ke^2}{r_2}+
               \frac{ke^2}{r_{12}},
\]
with the electrons separated at a distance 
$r_{12}=|\bm{r}_1-\bm{r}_2|$.
% --- end paragraph admon ---




% !split
\subsection{Quantum Monte Carlo: the helium atom}

% --- begin paragraph admon ---
\paragraph{}

The hamiltonian becomes then
\[
   \hat{H}=-\frac{\hbar^2\nabla_1^2}{2m}-\frac{\hbar^2\nabla_2^2}{2m}
          -\frac{2ke^2}{r_1}-\frac{2ke^2}{r_2}+
               \frac{ke^2}{r_{12}},
\]
and  Schroedingers equation reads
\[
   \hat{H}\psi=E\psi.
\]
All observables are evaluated with respect to the probability distribution
\[
   P(\bm{R})= \frac{\left|\psi_T(\bm{R})\right|^2}{\int \left|\psi_T(\bm{R})\right|^2d\bm{R}}.
\]
generated by the trial wave function.   
The trial wave function must approximate an exact 
eigenstate in order that accurate results are to be obtained.
% --- end paragraph admon ---



% !split
\subsection{Quantum Monte Carlo: the helium atom}

% --- begin paragraph admon ---
\paragraph{}
Choice of trial wave function for Helium:
Assume $r_1 \rightarrow 0$.
\[
   E_L(\bm{R})=\frac{1}{\psi_T(\bm{R})}H\psi_T(\bm{R})=
     \frac{1}{\psi_T(\bm{R})}\left(-\frac{1}{2}\nabla^2_1
     -\frac{Z}{r_1}\right)\psi_T(\bm{R}) + \mathrm{finite \hspace{0.1cm}terms}.
\]
\[ 
    E_L(R)=
    \frac{1}{{\cal R}_T(r_1)}\left(-\frac{1}{2}\frac{d^2}{dr_1^2}-
     \frac{1}{r_1}\frac{d}{dr_1}
     -\frac{Z}{r_1}\right){\cal R}_T(r_1) + \mathrm{finite\hspace{0.1cm} terms}
\]
For small values of $r_1$, the terms which dominate are
\[ 
    \lim_{r_1 \rightarrow 0}E_L(R)=
    \frac{1}{{\cal R}_T(r_1)}\left(-
     \frac{1}{r_1}\frac{d}{dr_1}
     -\frac{Z}{r_1}\right){\cal R}_T(r_1),
\]
since the second derivative does not diverge due to the finiteness of  $\Psi$ at the origin.
% --- end paragraph admon ---






% !split
\subsection{Quantum Monte Carlo: the helium atom}

% --- begin paragraph admon ---
\paragraph{}
This results in
\[
     \frac{1}{{\cal R}_T(r_1)}\frac{d {\cal R}_T(r_1)}{dr_1}=-Z,
\]
and
\[
   {\cal R}_T(r_1)\propto e^{-Zr_1}.
\]
A similar condition applies to electron 2 as well. 
For orbital momenta $l > 0$ we have 
\[
     \frac{1}{{\cal R}_T(r)}\frac{d {\cal R}_T(r)}{dr}=-\frac{Z}{l+1}.
\]
Similarly, studying the case $r_{12}\rightarrow 0$ we can write 
a possible trial wave function as
\[
   \psi_T(\bm{R})=e^{-\alpha(r_1+r_2)}e^{\beta r_{12}}.
    \label{eq:wavehelium2}
\]
The last equation can be generalized to
\[
   \psi_T(\bm{R})=\phi(\bm{r}_1)\phi(\bm{r}_2)\dots\phi(\bm{r}_N)
                   \prod_{i < j}f(r_{ij}),
\]
for a system with $N$ electrons or particles.
% --- end paragraph admon ---






% !split
\subsection{The first attempt at solving the helium atom}

% --- begin paragraph admon ---
\paragraph{}

During the development of our code we need to make several checks. It is also very instructive to compute a closed form expression for the local energy. Since our wave function is rather simple  it is straightforward
to find an analytic expressions.  Consider first the case of the simple helium function 
\[
   \Psi_T(\bm{r}_1,\bm{r}_2) = e^{-\alpha(r_1+r_2)}
\]
The local energy is for this case 
\[ 
E_{L1} = \left(\alpha-Z\right)\left(\frac{1}{r_1}+\frac{1}{r_2}\right)+\frac{1}{r_{12}}-\alpha^2
\]
which gives an expectation value for the local energy given by
\[
\langle E_{L1} \rangle = \alpha^2-2\alpha\left(Z-\frac{5}{16}\right)
\]
% --- end paragraph admon ---



% !split
\subsection{The first attempt at solving the Helium atom}

% --- begin paragraph admon ---
\paragraph{}

With closed form formulae we  can speed up the computation of the correlation. In our case
we write it as 
\[
\Psi_C= \exp{\left\{\sum_{i < j}\frac{ar_{ij}}{1+\beta r_{ij}}\right\}},
\]
which means that the gradient needed for the so-called quantum force and local energy 
can be calculated analytically.
This will speed up your code since the computation of the correlation part and the Slater determinant are the most 
time consuming parts in your code.  

We will refer to this correlation function as $\Psi_C$ or the \emph{linear Pade-Jastrow}.
% --- end paragraph admon ---



% !split
\subsection{The first attempt at solving the Helium atom}

% --- begin paragraph admon ---
\paragraph{}

We can test this by computing the local energy for our helium wave function
\[
   \psi_{T}(\bm{r}_1,\bm{r}_2) = 
   \exp{\left(-\alpha(r_1+r_2)\right)}
   \exp{\left(\frac{r_{12}}{2(1+\beta r_{12})}\right)}, 
\]
with $\alpha$ and $\beta$ as variational parameters.

The local energy is for this case 
\[ 
E_{L2} = E_{L1}+\frac{1}{2(1+\beta r_{12})^2}\left\{\frac{\alpha(r_1+r_2)}{r_{12}}(1-\frac{\bm{r}_1\bm{r}_2}{r_1r_2})-\frac{1}{2(1+\beta r_{12})^2}-\frac{2}{r_{12}}+\frac{2\beta}{1+\beta r_{12}}\right\}
\]
It is very useful to test your code against these expressions. It means also that you don't need to
compute a derivative numerically as discussed in the code example below.
% --- end paragraph admon ---



% !split
\subsection{The first attempt at solving the Helium atom}

% --- begin paragraph admon ---
\paragraph{}
For the computation of various derivatives with different types of wave functions, you will find it useful to use python with symbolic python, that is sympy, see URL: 'http://docs.sympy.org/latest/index.html'.  Using sympy allows you autogenerate both Latex code as well c++, python or Fortran codes. Here you will find some simple examples. We choose 
the $2s$ hydrogen-orbital  (not normalized) as an example
\[
 \phi_{2s}(\bm{r}) = (Zr - 2)\exp{-(\frac{1}{2}Zr)},
\]
with $ r^2 = x^2 + y^2 + z^2$.

\bpycod
from sympy import symbols, diff, exp, sqrt
x, y, z, Z = symbols('x y z Z')
r = sqrt(x*x + y*y + z*z)
r
phi = (Z*r - 2)*exp(-Z*r/2)
phi
diff(phi, x)
\epycod
This doesn't look very nice, but sympy provides several functions that allow for improving and simplifying the output.
% --- end paragraph admon ---



% !split
\subsection{The first attempt at solving the Helium atom}

% --- begin paragraph admon ---
\paragraph{}
We can improve our output by factorizing and substituting expressions
\bpycod
from sympy import symbols, diff, exp, sqrt, factor, Symbol, printing
x, y, z, Z = symbols('x y z Z')
r = sqrt(x*x + y*y + z*z)
phi = (Z*r - 2)*exp(-Z*r/2)
R = Symbol('r') #Creates a symbolic equivalent of r
#print latex and c++ code
print printing.latex(diff(phi, x).factor().subs(r, R))
print printing.ccode(diff(phi, x).factor().subs(r, R))
\epycod
% --- end paragraph admon ---




% !split
\subsection{The first attempt at solving the Helium atom}

% --- begin paragraph admon ---
\paragraph{}
We can in turn look at second derivatives
\bpycod
from sympy import symbols, diff, exp, sqrt, factor, Symbol, printing
x, y, z, Z = symbols('x y z Z')
r = sqrt(x*x + y*y + z*z)
phi = (Z*r - 2)*exp(-Z*r/2)
R = Symbol('r') #Creates a symbolic equivalent of r
(diff(diff(phi, x), x) + diff(diff(phi, y), y) + diff(diff(phi, z), z)).factor().subs(r, R)
# Collect the Z values
(diff(diff(phi, x), x) + diff(diff(phi, y), y) +diff(diff(phi, z), z)).factor().collect(Z).subs(r, R)
# Factorize also the r**2 terms
(diff(diff(phi, x), x) + diff(diff(phi, y), y) + diff(diff(phi, z), z)).factor().collect(Z).subs(r, R).subs(r**2, R**2).factor()
print printing.ccode((diff(diff(phi, x), x) + diff(diff(phi, y), y) + diff(diff(phi, z), z)).factor().collect(Z).subs(r, R).subs(r**2, R**2).factor())
\epycod
With some practice this allows one to be able to check one's own calculation and translate automatically into code lines.
% --- end paragraph admon ---



% !split
\subsection{The first attempt at solving the Helium atom}

% --- begin paragraph admon ---
\paragraph{The c++ code with a VMC Solver class, main program first.}
\bcppcod
#include "vmcsolver.h"
#include <iostream>
using namespace std;

int main()
{
    VMCSolver *solver = new VMCSolver();
    solver->runMonteCarloIntegration();
    return 0;
}
\ecppcod
% --- end paragraph admon ---



% !split
\subsection{The first attempt at solving the Helium atom}

% --- begin paragraph admon ---
\paragraph{The c++ code with a VMC Solver class, the VMCSolver header file.}
\bcppcod
#ifndef VMCSOLVER_H
#define VMCSOLVER_H
#include <armadillo>
using namespace arma;
class VMCSolver
{
public:
    VMCSolver();
    void runMonteCarloIntegration();

private:
    double waveFunction(const mat &r);
    double localEnergy(const mat &r);
    int nDimensions;
    int charge;
    double stepLength;
    int nParticles;
    double h;
    double h2;
    long idum;
    double alpha;
    int nCycles;
    mat rOld;
    mat rNew;
};
#endif // VMCSOLVER_H
\ecppcod
% --- end paragraph admon ---



% !split
\subsection{The first attempt at solving the Helium atom}

% --- begin paragraph admon ---
\paragraph{The c++ code with a VMC Solver class, VMCSolver codes, initialize.}
\bcppcod
#include "vmcsolver.h"
#include "lib.h"
#include <armadillo>
#include <iostream>
using namespace arma;
using namespace std;

VMCSolver::VMCSolver() :
    nDimensions(3),
    charge(2),
    stepLength(1.0),
    nParticles(2),
    h(0.001),
    h2(1000000),
    idum(-1),
    alpha(0.5*charge),
    nCycles(1000000)
{
}
\ecppcod
% --- end paragraph admon ---




% !split
\subsection{The first attempt at solving the Helium atom}

% --- begin paragraph admon ---
\paragraph{The c++ code with a VMC Solver class, VMCSolver codes.}
\bcppcod
void VMCSolver::runMonteCarloIntegration()
{
    rOld = zeros<mat>(nParticles, nDimensions);
    rNew = zeros<mat>(nParticles, nDimensions);
    double waveFunctionOld = 0;
    double waveFunctionNew = 0;
    double energySum = 0;
    double energySquaredSum = 0;
    double deltaE;
    // initial trial positions
    for(int i = 0; i < nParticles; i++) {
        for(int j = 0; j < nDimensions; j++) {
            rOld(i,j) = stepLength * (ran2(&idum) - 0.5);
        }
    }
    rNew = rOld;
    // loop over Monte Carlo cycles
    for(int cycle = 0; cycle < nCycles; cycle++) {
        // Store the current value of the wave function
        waveFunctionOld = waveFunction(rOld);
        // New position to test
        for(int i = 0; i < nParticles; i++) {
            for(int j = 0; j < nDimensions; j++) {
                rNew(i,j) = rOld(i,j) + stepLength*(ran2(&idum) - 0.5);
            }
            // Recalculate the value of the wave function
            waveFunctionNew = waveFunction(rNew);
            // Check for step acceptance (if yes, update position, if no, reset position)
            if(ran2(&idum) <= (waveFunctionNew*waveFunctionNew) / (waveFunctionOld*waveFunctionOld)) {
                for(int j = 0; j < nDimensions; j++) {
                    rOld(i,j) = rNew(i,j);
                    waveFunctionOld = waveFunctionNew;
                }
            } else {
                for(int j = 0; j < nDimensions; j++) {
                    rNew(i,j) = rOld(i,j);
                }
            }
            // update energies
            deltaE = localEnergy(rNew);
            energySum += deltaE;
            energySquaredSum += deltaE*deltaE;
        }
    }
    double energy = energySum/(nCycles * nParticles);
    double energySquared = energySquaredSum/(nCycles * nParticles);
    cout << "Energy: " << energy << " Energy (squared sum): " << energySquared << endl;
}
\ecppcod
% --- end paragraph admon ---




% !split
\subsection{The first attempt at solving the Helium atom}

% --- begin paragraph admon ---
\paragraph{The c++ code with a VMC Solver class, VMCSolver codes.}
\bcppcod
double VMCSolver::localEnergy(const mat &r)
{
    mat rPlus = zeros<mat>(nParticles, nDimensions);
    mat rMinus = zeros<mat>(nParticles, nDimensions);
    rPlus = rMinus = r;
    double waveFunctionMinus = 0;
    double waveFunctionPlus = 0;
    double waveFunctionCurrent = waveFunction(r);
    // Kinetic energy, brute force derivations
    double kineticEnergy = 0;
    for(int i = 0; i < nParticles; i++) {
        for(int j = 0; j < nDimensions; j++) {
            rPlus(i,j) += h;
            rMinus(i,j) -= h;
            waveFunctionMinus = waveFunction(rMinus);
            waveFunctionPlus = waveFunction(rPlus);
            kineticEnergy -= (waveFunctionMinus + waveFunctionPlus - 2 * waveFunctionCurrent);
            rPlus(i,j) = r(i,j);
            rMinus(i,j) = r(i,j);
        }
    }
    kineticEnergy = 0.5 * h2 * kineticEnergy / waveFunctionCurrent;
    // Potential energy
    double potentialEnergy = 0;
    double rSingleParticle = 0;
    for(int i = 0; i < nParticles; i++) {
        rSingleParticle = 0;
        for(int j = 0; j < nDimensions; j++) {
            rSingleParticle += r(i,j)*r(i,j);
        }
        potentialEnergy -= charge / sqrt(rSingleParticle);
    }
    // Contribution from electron-electron potential
    double r12 = 0;
    for(int i = 0; i < nParticles; i++) {
        for(int j = i + 1; j < nParticles; j++) {
            r12 = 0;
            for(int k = 0; k < nDimensions; k++) {
                r12 += (r(i,k) - r(j,k)) * (r(i,k) - r(j,k));
            }
            potentialEnergy += 1 / sqrt(r12);
        }
    }
    return kineticEnergy + potentialEnergy;
}
\ecppcod
% --- end paragraph admon ---




% !split
\subsection{The first attempt at solving the Helium atom}

% --- begin paragraph admon ---
\paragraph{The c++ code with a VMC Solver class, VMCSolver codes.}
\bcppcod
double VMCSolver::waveFunction(const mat &r)
{
    double argument = 0;
    for(int i = 0; i < nParticles; i++) {
        double rSingleParticle = 0;
        for(int j = 0; j < nDimensions; j++) {
            rSingleParticle += r(i,j) * r(i,j);
        }
        argument += sqrt(rSingleParticle);
    }
    return exp(-argument * alpha);
}
\ecppcod
% --- end paragraph admon ---




% !split
\subsection{The first attempt at solving the Helium atom}

% --- begin paragraph admon ---
\paragraph{The c++ code with a VMC Solver class, all in one file.}
\bcppcod
\ecppcod
% --- end paragraph admon ---



% !split
\subsection{The first attempt at solving the Helium atom}

% --- begin paragraph admon ---
\paragraph{The c++ code with a VMC Solver class, the VMCSolver header file.}
\bcppcod
#include <armadillo>
#include <iostream>
using namespace arma;
using namespace std;
double ran2(long *);

class VMCSolver
{
public:
    VMCSolver();
    void runMonteCarloIntegration();

private:
    double waveFunction(const mat &r);
    double localEnergy(const mat &r);
    int nDimensions;
    int charge;
    double stepLength;
    int nParticles;
    double h;
    double h2;
    long idum;
    double alpha;
    int nCycles;
    mat rOld;
    mat rNew;
};

VMCSolver::VMCSolver() :
    nDimensions(3),
    charge(2),
    stepLength(1.0),
    nParticles(2),
    h(0.001),
    h2(1000000),
    idum(-1),
    alpha(0.5*charge),
    nCycles(1000000)
{
}

void VMCSolver::runMonteCarloIntegration()
{
    rOld = zeros<mat>(nParticles, nDimensions);
    rNew = zeros<mat>(nParticles, nDimensions);
    double waveFunctionOld = 0;
    double waveFunctionNew = 0;
    double energySum = 0;
    double energySquaredSum = 0;
    double deltaE;
    // initial trial positions
    for(int i = 0; i < nParticles; i++) {
        for(int j = 0; j < nDimensions; j++) {
            rOld(i,j) = stepLength * (ran2(&idum) - 0.5);
        }
    }
    rNew = rOld;
    // loop over Monte Carlo cycles
    for(int cycle = 0; cycle < nCycles; cycle++) {
        // Store the current value of the wave function
        waveFunctionOld = waveFunction(rOld);
        // New position to test
        for(int i = 0; i < nParticles; i++) {
            for(int j = 0; j < nDimensions; j++) {
                rNew(i,j) = rOld(i,j) + stepLength*(ran2(&idum) - 0.5);
            }
            // Recalculate the value of the wave function
            waveFunctionNew = waveFunction(rNew);
            // Check for step acceptance (if yes, update position, if no, reset position)
            if(ran2(&idum) <= (waveFunctionNew*waveFunctionNew) / (waveFunctionOld*waveFunctionOld)) {
                for(int j = 0; j < nDimensions; j++) {
                    rOld(i,j) = rNew(i,j);
                    waveFunctionOld = waveFunctionNew;
                }
            } else {
                for(int j = 0; j < nDimensions; j++) {
                    rNew(i,j) = rOld(i,j);
                }
            }
            // update energies
            deltaE = localEnergy(rNew);
            energySum += deltaE;
            energySquaredSum += deltaE*deltaE;
        }
    }
    double energy = energySum/(nCycles * nParticles);
    double energySquared = energySquaredSum/(nCycles * nParticles);
    cout << "Energy: " << energy << " Energy (squared sum): " << energySquared << endl;
}

double VMCSolver::localEnergy(const mat &r)
{
    mat rPlus = zeros<mat>(nParticles, nDimensions);
    mat rMinus = zeros<mat>(nParticles, nDimensions);
    rPlus = rMinus = r;
    double waveFunctionMinus = 0;
    double waveFunctionPlus = 0;
    double waveFunctionCurrent = waveFunction(r);
    // Kinetic energy, brute force derivations
    double kineticEnergy = 0;
    for(int i = 0; i < nParticles; i++) {
        for(int j = 0; j < nDimensions; j++) {
            rPlus(i,j) += h;
            rMinus(i,j) -= h;
            waveFunctionMinus = waveFunction(rMinus);
            waveFunctionPlus = waveFunction(rPlus);
            kineticEnergy -= (waveFunctionMinus + waveFunctionPlus - 2 * waveFunctionCurrent);
            rPlus(i,j) = r(i,j);
            rMinus(i,j) = r(i,j);
        }
    }
    kineticEnergy = 0.5 * h2 * kineticEnergy / waveFunctionCurrent;
    // Potential energy
    double potentialEnergy = 0;
    double rSingleParticle = 0;
    for(int i = 0; i < nParticles; i++) {
        rSingleParticle = 0;
        for(int j = 0; j < nDimensions; j++) {
            rSingleParticle += r(i,j)*r(i,j);
        }
        potentialEnergy -= charge / sqrt(rSingleParticle);
    }
    // Contribution from electron-electron potential
    double r12 = 0;
    for(int i = 0; i < nParticles; i++) {
        for(int j = i + 1; j < nParticles; j++) {
            r12 = 0;
            for(int k = 0; k < nDimensions; k++) {
                r12 += (r(i,k) - r(j,k)) * (r(i,k) - r(j,k));
            }
            potentialEnergy += 1 / sqrt(r12);
        }
    }
    return kineticEnergy + potentialEnergy;
}

double VMCSolver::waveFunction(const mat &r)
{
    double argument = 0;
    for(int i = 0; i < nParticles; i++) {
        double rSingleParticle = 0;
        for(int j = 0; j < nDimensions; j++) {
            rSingleParticle += r(i,j) * r(i,j);
        }
        argument += sqrt(rSingleParticle);
    }
    return exp(-argument * alpha);
}

/*
** The function
**         ran2()
** is a long periode (> 2 x 10^18) random number generator of
** L'Ecuyer and Bays-Durham shuffle and added safeguards.
** Call with idum a negative integer to initialize; thereafter,
** do not alter idum between sucessive deviates in a
** sequence. RNMX should approximate the largest floating point value
** that is less than 1.
** The function returns a uniform deviate between 0.0 and 1.0
** (exclusive of end-point values).
*/

#define IM1 2147483563
#define IM2 2147483399
#define AM (1.0/IM1)
#define IMM1 (IM1-1)
#define IA1 40014
#define IA2 40692
#define IQ1 53668
#define IQ2 52774
#define IR1 12211
#define IR2 3791
#define NTAB 32
#define NDIV (1+IMM1/NTAB)
#define EPS 1.2e-7
#define RNMX (1.0-EPS)

double ran2(long *idum)
{
  int            j;
  long           k;
  static long    idum2 = 123456789;
  static long    iy=0;
  static long    iv[NTAB];
  double         temp;

  if(*idum <= 0) {
    if(-(*idum) < 1) *idum = 1;
    else             *idum = -(*idum);
    idum2 = (*idum);
    for(j = NTAB + 7; j >= 0; j--) {
      k     = (*idum)/IQ1;
      *idum = IA1*(*idum - k*IQ1) - k*IR1;
      if(*idum < 0) *idum +=  IM1;
      if(j < NTAB)  iv[j]  = *idum;
    }
    iy=iv[0];
  }
  k     = (*idum)/IQ1;
  *idum = IA1*(*idum - k*IQ1) - k*IR1;
  if(*idum < 0) *idum += IM1;
  k     = idum2/IQ2;
  idum2 = IA2*(idum2 - k*IQ2) - k*IR2;
  if(idum2 < 0) idum2 += IM2;
  j     = iy/NDIV;
  iy    = iv[j] - idum2;
  iv[j] = *idum;
  if(iy < 1) iy += IMM1;
  if((temp = AM*iy) > RNMX) return RNMX;
  else return temp;
}
#undef IM1
#undef IM2
#undef AM
#undef IMM1
#undef IA1
#undef IA2
#undef IQ1
#undef IQ2
#undef IR1
#undef IR2
#undef NTAB
#undef NDIV
#undef EPS
#undef RNMX

// End: function ran2()


#include <iostream>
using namespace std;

int main()
{
    VMCSolver *solver = new VMCSolver();
    solver->runMonteCarloIntegration();
    return 0;
}
\ecppcod
% --- end paragraph admon ---




% !split
\subsection{The first attempt at solving the Helium atom}

% --- begin paragraph admon ---
\paragraph{Exercises for first lab session.}

\begin{itemize}
 \item If you have never used git, Qt, armadillo etc, get familiar with them

 \item Study the simple program in the folder URL: 'https://github.com/CompPhysics/ComputationalPhysics2/tree/gh-pages/doc/pub/vmc/programs/'

 \item Implement the closed form expression for the local energy and the so-called quantum force

 \item Convince yourself that the closed form expressions are correct. Check both wave functions

 \item Implement the closed form expression for the local energy and compare with a code where the second derivatives are computed numerically.
\end{itemize}

\noindent
% --- end paragraph admon ---










% ------------------- end of main content ---------------


% #ifdef PREAMBLE
\printindex

\end{document}
% #endif

