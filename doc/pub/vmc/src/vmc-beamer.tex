
% LaTeX Beamer file automatically generated from DocOnce
% https://github.com/hplgit/doconce

%-------------------- begin beamer-specific preamble ----------------------

\documentclass{beamer}

\usetheme{red_plain}
\usecolortheme{default}

% turn off the almost invisible, yet disturbing, navigation symbols:
\setbeamertemplate{navigation symbols}{}

% Examples on customization:
%\usecolortheme[named=RawSienna]{structure}
%\usetheme[height=7mm]{Rochester}
%\setbeamerfont{frametitle}{family=\rmfamily,shape=\itshape}
%\setbeamertemplate{items}[ball]
%\setbeamertemplate{blocks}[rounded][shadow=true]
%\useoutertheme{infolines}
%
%\usefonttheme{}
%\useinntertheme{}
%
%\setbeameroption{show notes}
%\setbeameroption{show notes on second screen=right}

% fine for B/W printing:
%\usecolortheme{seahorse}

\usepackage{pgf,pgfarrows,pgfnodes,pgfautomata,pgfheaps,pgfshade}
\usepackage{graphicx}
\usepackage{epsfig}
\usepackage{relsize}

\usepackage{fancybox}  % make sure fancybox is loaded before fancyvrb

\usepackage{fancyvrb}
\usepackage{minted} % requires pygments and latex -shell-escape filename
%\usepackage{anslistings}

\usepackage{amsmath,amssymb,bm}
%\usepackage[latin1]{inputenc}
\usepackage[T1]{fontenc}
\usepackage[utf8]{inputenc}
\usepackage{colortbl}
\usepackage[english]{babel}
\usepackage{tikz}
\usepackage{framed}
% Use some nice templates
\beamertemplatetransparentcovereddynamic

% --- begin table of contents based on sections ---
% Delete this, if you do not want the table of contents to pop up at
% the beginning of each section:
% (Only section headings can enter the table of contents in Beamer
% slides generated from DocOnce source, while subsections are used
% for the title in ordinary slides.)
\AtBeginSection[]
{
  \begin{frame}<beamer>[plain]
  \frametitle{}
  %\frametitle{Outline}
  \tableofcontents[currentsection]
  \end{frame}
}
% --- end table of contents based on sections ---

% If you wish to uncover everything in a step-wise fashion, uncomment
% the following command:

%\beamerdefaultoverlayspecification{<+->}

\newcommand{\shortinlinecomment}[3]{\note{\textbf{#1}: #2}}
\newcommand{\longinlinecomment}[3]{\shortinlinecomment{#1}{#2}{#3}}

\definecolor{linkcolor}{rgb}{0,0,0.4}
\hypersetup{
    colorlinks=true,
    linkcolor=linkcolor,
    urlcolor=linkcolor,
    pdfmenubar=true,
    pdftoolbar=true,
    bookmarksdepth=3
    }
\setlength{\parskip}{0pt}  % {1em}

\newenvironment{doconceexercise}{}{}
\newcounter{doconceexercisecounter}
\newenvironment{doconce:movie}{}{}
\newcounter{doconce:movie:counter}

\newcommand{\subex}[1]{\noindent\textbf{#1}}  % for subexercises: a), b), etc

%-------------------- end beamer-specific preamble ----------------------

% Add user's preamble




% insert custom LaTeX commands...

\raggedbottom
\makeindex

%-------------------- end preamble ----------------------

\begin{document}




% ------------------- main content ----------------------



% ----------------- title -------------------------

\title{Slides from FYS4411/9411 Variational Monte Carlo methods}

% ----------------- author(s) -------------------------

\author{Morten Hjorth-Jensen\inst{1,2}}
\institute{Department of Physics, University of Oslo, Oslo, Norway\inst{1}
\and
Department of Physics and Astronomy and National Superconducting Cyclotron Laboratory, Michigan State University, East Lansing, MI 48824, USA\inst{2}}
% ----------------- end author(s) -------------------------

\date{Spring 2015
% <optional titlepage figure>
}

\begin{frame}[plain,fragile]
\titlepage
\end{frame}

\begin{frame}[plain,fragile]
\frametitle{Quantum Monte Carlo Motivation}

\begin{block}{}
Given a hamiltonian $H$ and a trial wave function $\Psi_T$, the variational principle states that the expectation value of $\langle H \rangle$, defined through 
\[
   E[H]= \langle H \rangle =
   \frac{\int d\bm{R}\Psi^{\ast}_T(\bm{R})H(\bm{R})\Psi_T(\bm{R})}
        {\int d\bm{R}\Psi^{\ast}_T(\bm{R})\Psi_T(\bm{R})},
\]
is an upper bound to the ground state energy $E_0$ of the hamiltonian $H$, that is 
\[
    E_0 \le \langle H \rangle .
\]
In general, the integrals involved in the calculation of various  expectation values  are multi-dimensional ones. Traditional integration methods such as the Gauss-Legendre will not be adequate for say the  computation of the energy of a many-body system.
\end{block}
\end{frame}

\begin{frame}[plain,fragile]
\frametitle{Quantum Monte Carlo Motivation}

\begin{block}{}
The trial wave function can be expanded in the eigenstates of the hamiltonian since they form a complete set, viz.,
\[
   \Psi_T(\bm{R})=\sum_i a_i\Psi_i(\bm{R}),
\]
and assuming the set of eigenfunctions to be normalized one obtains 
\[
     \frac{\sum_{nm}a^*_ma_n \int d\bm{R}\Psi^{\ast}_m(\bm{R})H(\bm{R})\Psi_n(\bm{R})}
        {\sum_{nm}a^*_ma_n \int d\bm{R}\Psi^{\ast}_m(\bm{R})\Psi_n(\bm{R})} =\frac{\sum_{n}a^2_n E_n}
        {\sum_{n}a^2_n} \ge E_0,
\]
where we used that $H(\bm{R})\Psi_n(\bm{R})=E_n\Psi_n(\bm{R})$.
In general, the integrals involved in the calculation of various  expectation
values  are multi-dimensional ones. 
The variational principle yields the lowest state of a given symmetry.

\end{block}
\end{frame}

\begin{frame}[plain,fragile]
\frametitle{Quantum Monte Carlo Motivation}

\begin{block}{}
In most cases, a wave function has only small values in large parts of 
configuration space, and a straightforward procedure which uses
homogenously distributed random points in configuration space 
will most likely lead to poor results. This may suggest that some kind
of importance sampling combined with e.g., the Metropolis algorithm 
may be  a more efficient way of obtaining the ground state energy.
The hope is then that those regions of configurations space where
the wave function assumes appreciable values are sampled more 
efficiently. 
\end{block}
\end{frame}

\begin{frame}[plain,fragile]
\frametitle{Quantum Monte Carlo Motivation}

\begin{block}{}
The tedious part in a VMC calculation is the search for the variational
minimum. A good knowledge of the system is required in order to carry out
reasonable VMC calculations. This is not always the case, 
and often VMC calculations 
serve rather as the starting
point for so-called diffusion Monte Carlo calculations (DMC). DMC is a way of
solving exactly the many-body Schroedinger equation by means of 
a stochastic procedure. A good guess on the binding energy
and its wave function is however necessary. 
A carefully performed VMC calculation can aid in this context. 
\end{block}
\end{frame}

\begin{frame}[plain,fragile]
\frametitle{Quantum Monte Carlo Motivation}

\begin{block}{}
\begin{itemize}
\item Construct first a trial wave function $\psi_T(\bm{R},\bm{\alpha})$,  for a many-body system consisting of $N$ particles located at positions 
\end{itemize}

\noindent
$\bm{R}=(\bm{R}_1,\dots ,\bm{R}_N)$. The trial wave function depends on $\alpha$ variational parameters $\bm{\alpha}=(\alpha_1,\dots ,\alpha_M)$.
\begin{itemize}
\item Then we evaluate the expectation value of the hamiltonian $H$ 
\end{itemize}

\noindent
\[
   E[H]=\langle H \rangle =
   \frac{\int d\bm{R}\Psi^{\ast}_{T}(\bm{R},\bm{\alpha})H(\bm{R})\Psi_{T}(\bm{R},\bm{\alpha})}
        {\int d\bm{R}\Psi^{\ast}_{T}(\bm{R},\bm{\alpha})\Psi_{T}(\bm{R},\bm{\alpha})}.
\]
\begin{itemize}
\item Thereafter we vary $\alpha$ according to some minimization algorithm and return to the first step.
\end{itemize}

\noindent
\end{block}
\end{frame}

\begin{frame}[plain,fragile]
\frametitle{Quantum Monte Carlo Motivation}

\begin{block}{Basic steps }
Choose a trial wave function
$\psi_T(\bm{R})$.
\[
   P(\bm{R})= \frac{\left|\psi_T(\bm{R})\right|^2}{\int \left|\psi_T(\bm{R})\right|^2d\bm{R}}.
\]
This is our new probability distribution function  (PDF).
The approximation to the expectation value of the Hamiltonian is now 
\[
   E[H(\bm{\alpha})] = 
   \frac{\int d\bm{R}\Psi^{\ast}_T(\bm{R},\bm{\alpha})H(\bm{R})\Psi_T(\bm{R},\bm{\alpha})}
        {\int d\bm{R}\Psi^{\ast}_T(\bm{R},\bm{\alpha})\Psi_T(\bm{R},\bm{\alpha})}.
\]
\end{block}
\end{frame}

\begin{frame}[plain,fragile]
\frametitle{Quantum Monte Carlo Motivation}

\begin{block}{}
Define a new quantity
\[
   E_L(\bm{R},\bm{\alpha})=\frac{1}{\psi_T(\bm{R},\bm{\alpha})}H\psi_T(\bm{R},\bm{\alpha}),
   \label{eq:locale1}
\]
called the local energy, which, together with our trial PDF yields
\[
  E[H(\bm{\alpha})]= = \int P(\bm{R})E_L(\bm{R}) d\bm{R}\approx \frac{1}{N}\sum_{i=1}^NP(\bm{R_i},\bm{\alpha})E_L(\bm{R_i},\bm{\alpha})
  \label{eq:vmc1}
\]
with $N$ being the number of Monte Carlo samples.
\end{block}
\end{frame}

\begin{frame}[plain,fragile]
\frametitle{Quantum Monte Carlo}

\begin{block}{}
The Algorithm for performing a variational Monte Carlo calculations runs thus as this

\begin{itemize}
   \item Initialisation: Fix the number of Monte Carlo steps. Choose an initial $\bm{R}$ and variational parameters $\alpha$ and calculate $\left|\psi_T^{\alpha}(\bm{R})\right|^2$. 

   \item Initialise the energy and the variance and start the Monte Carlo calculation.
\begin{itemize}

      \item Calculate  a trial position  $\bm{R}_p=\bm{R}+r*step$ where $r$ is a random variable $r \in [0,1]$.

      \item Metropolis algorithm to accept or reject this move  $w = P(\bm{R}_p)/P(\bm{R})$.

      \item If the step is accepted, then we set $\bm{R}=\bm{R}_p$. 

      \item Update averages

\end{itemize}

\noindent
   \item Finish and compute final averages.
\end{itemize}

\noindent
Observe that the jumping in space is governed by the variable \emph{step}. This is Called brute-force sampling.
Need importance sampling to get more relevant sampling, see lectures below.
\end{block}
\end{frame}

\begin{frame}[plain,fragile]
\frametitle{Quantum Monte Carlo: hydrogen atom}

\begin{block}{}
The radial Schroedinger equation for the hydrogen atom can be
written as
\[
-\frac{\hbar^2}{2m}\frac{\partial^2 u(r)}{\partial r^2}-
\left(\frac{ke^2}{r}-\frac{\hbar^2l(l+1)}{2mr^2}\right)u(r)=Eu(r),
\]
or with dimensionless variables
\[
-\frac{1}{2}\frac{\partial^2 u(\rho)}{\partial \rho^2}-
\frac{u(\rho)}{\rho}+\frac{l(l+1)}{2\rho^2}u(\rho)-\lambda u(\rho)=0,
\label{eq:hydrodimless1}
\]
with the hamiltonian
\[
H=-\frac{1}{2}\frac{\partial^2 }{\partial \rho^2}-
\frac{1}{\rho}+\frac{l(l+1)}{2\rho^2}.
\]
Use variational parameter $\alpha$ in the trial
wave function 
\[
   u_T^{\alpha}(\rho)=\alpha\rho e^{-\alpha\rho}. 
   \label{eq:trialhydrogen}
\]
\end{block}
\end{frame}

\begin{frame}[plain,fragile]
\frametitle{Quantum Monte Carlo: hydrogen atom}

\begin{block}{}
Inserting this wave function into the expression for the
local energy $E_L$ gives
\[
   E_L(\rho)=-\frac{1}{\rho}-
              \frac{\alpha}{2}\left(\alpha-\frac{2}{\rho}\right).
\]
A simple variational Monte Carlo calculation results in

{\footnotesize
\begin{tabular}{cccc}
\hline
\multicolumn{1}{c}{ $\alpha$ } & \multicolumn{1}{c}{ $\langle H \rangle $ } & \multicolumn{1}{c}{ $\sigma^2$ } & \multicolumn{1}{c}{ $\sigma/\sqrt{N}$ } \\
\hline
7.00000E-01 & -4.57759E-01         & 4.51201E-02 & 6.71715E-04       \\
8.00000E-01 & -4.81461E-01         & 3.05736E-02 & 5.52934E-04       \\
9.00000E-01 & -4.95899E-01         & 8.20497E-03 & 2.86443E-04       \\
1.00000E-00 & -5.00000E-01         & 0.00000E+00 & 0.00000E+00       \\
1.10000E+00 & -4.93738E-01         & 1.16989E-02 & 3.42036E-04       \\
1.20000E+00 & -4.75563E-01         & 8.85899E-02 & 9.41222E-04       \\
1.30000E+00 & -4.54341E-01         & 1.45171E-01 & 1.20487E-03       \\
\hline
\end{tabular}
}

\noindent
\end{block}
\end{frame}

\begin{frame}[plain,fragile]
\frametitle{Quantum Monte Carlo: hydrogen atom}

\begin{block}{}

We note that at $\alpha=1$ we obtain the exact
result, and the variance is zero, as it should. The reason is that 
we then have the exact wave function, and the action of the hamiltionan
on the wave function
\[
   H\psi = \mathrm{constant}\times \psi,
\]
yields just a constant. The integral which defines various 
expectation values involving moments of the hamiltonian becomes then
\[
   \langle H^n \rangle =
   \frac{\int d\bm{R}\Psi^{\ast}_T(\bm{R})H^n(\bm{R})\Psi_T(\bm{R})}
        {\int d\bm{R}\Psi^{\ast}_T(\bm{R})\Psi_T(\bm{R})}=
\mathrm{constant}\times\frac{\int d\bm{R}\Psi^{\ast}_T(\bm{R})\Psi_T(\bm{R})}
        {\int d\bm{R}\Psi^{\ast}_T(\bm{R})\Psi_T(\bm{R})}=\mathrm{constant}.
\]
\textbf{This gives an important information: the exact wave function leads to zero variance!}
Variation is then performed by minimizing both the energy and the variance.
\end{block}
\end{frame}

\begin{frame}[plain,fragile]
\frametitle{Quantum Monte Carlo: the helium atom}

\begin{block}{}
The helium atom consists of two electrons and a nucleus with
charge $Z=2$. 
The contribution  
to the potential energy due to the attraction from the nucleus is
\[
   -\frac{2ke^2}{r_1}-\frac{2ke^2}{r_2},
\] 
and if we add the repulsion arising from the two 
interacting electrons, we obtain the potential energy
\[
 V(r_1, r_2)=-\frac{2ke^2}{r_1}-\frac{2ke^2}{r_2}+
               \frac{ke^2}{r_{12}},
\]
with the electrons separated at a distance 
$r_{12}=|\bm{r}_1-\bm{r}_2|$.
\end{block}
\end{frame}

\begin{frame}[plain,fragile]
\frametitle{Quantum Monte Carlo: the helium atom}

\begin{block}{}

The hamiltonian becomes then
\[
   \hat{H}=-\frac{\hbar^2\nabla_1^2}{2m}-\frac{\hbar^2\nabla_2^2}{2m}
          -\frac{2ke^2}{r_1}-\frac{2ke^2}{r_2}+
               \frac{ke^2}{r_{12}},
\]
and  Schroedingers equation reads
\[
   \hat{H}\psi=E\psi.
\]
All observables are evaluated with respect to the probability distribution
\[
   P(\bm{R})= \frac{\left|\psi_T(\bm{R})\right|^2}{\int \left|\psi_T(\bm{R})\right|^2d\bm{R}}.
\]
generated by the trial wave function.   
The trial wave function must approximate an exact 
eigenstate in order that accurate results are to be obtained. 
\end{block}
\end{frame}

\begin{frame}[plain,fragile]
\frametitle{Quantum Monte Carlo: the helium atom}

\begin{block}{}
Choice of trial wave function for Helium:
Assume $r_1 \rightarrow 0$.
\[
   E_L(\bm{R})=\frac{1}{\psi_T(\bm{R})}H\psi_T(\bm{R})=
     \frac{1}{\psi_T(\bm{R})}\left(-\frac{1}{2}\nabla^2_1
     -\frac{Z}{r_1}\right)\psi_T(\bm{R}) + \mathrm{finite \hspace{0.1cm}terms}.
\]
\[ 
    E_L(R)=
    \frac{1}{{\cal R}_T(r_1)}\left(-\frac{1}{2}\frac{d^2}{dr_1^2}-
     \frac{1}{r_1}\frac{d}{dr_1}
     -\frac{Z}{r_1}\right){\cal R}_T(r_1) + \mathrm{finite\hspace{0.1cm} terms}
\]
For small values of $r_1$, the terms which dominate are
\[ 
    \lim_{r_1 \rightarrow 0}E_L(R)=
    \frac{1}{{\cal R}_T(r_1)}\left(-
     \frac{1}{r_1}\frac{d}{dr_1}
     -\frac{Z}{r_1}\right){\cal R}_T(r_1),
\]
since the second derivative does not diverge due to the finiteness of  $\Psi$ at the origin.
\end{block}
\end{frame}

\begin{frame}[plain,fragile]
\frametitle{Quantum Monte Carlo: the helium atom}

\begin{block}{}
This results in
\[
     \frac{1}{{\cal R}_T(r_1)}\frac{d {\cal R}_T(r_1)}{dr_1}=-Z,
\]
and
\[
   {\cal R}_T(r_1)\propto e^{-Zr_1}.
\]
A similar condition applies to electron 2 as well. 
For orbital momenta $l > 0$ we have 
\[
     \frac{1}{{\cal R}_T(r)}\frac{d {\cal R}_T(r)}{dr}=-\frac{Z}{l+1}.
\]
Similarly, studying the case $r_{12}\rightarrow 0$ we can write 
a possible trial wave function as
\[
   \psi_T(\bm{R})=e^{-\alpha(r_1+r_2)}e^{\beta r_{12}}.
    \label{eq:wavehelium2}
\]
The last equation can be generalized to
\[
   \psi_T(\bm{R})=\phi(\bm{r}_1)\phi(\bm{r}_2)\dots\phi(\bm{r}_N)
                   \prod_{i < j}f(r_{ij}),
\]
for a system with $N$ electrons or particles. 
\end{block}
\end{frame}

\begin{frame}[plain,fragile]
\frametitle{The first attempt at solving the helium atom}

\begin{block}{}

During the development of our code we need to make several checks. It is also very instructive to compute a closed form expression for the local energy. Since our wave function is rather simple  it is straightforward
to find an analytic expressions.  Consider first the case of the simple helium function 
\[
   \Psi_T(\bm{r}_1,\bm{r}_2) = e^{-\alpha(r_1+r_2)}
\]
The local energy is for this case 
\[ 
E_{L1} = \left(\alpha-Z\right)\left(\frac{1}{r_1}+\frac{1}{r_2}\right)+\frac{1}{r_{12}}-\alpha^2
\]
which gives an expectation value for the local energy given by
\[
\langle E_{L1} \rangle = \alpha^2-2\alpha\left(Z-\frac{5}{16}\right)
\]
\end{block}
\end{frame}

\begin{frame}[plain,fragile]
\frametitle{The first attempt at solving the Helium atom}

\begin{block}{}

With closed form formulae we  can speed up the computation of the correlation. In our case
we write it as 
\[
\Psi_C= \exp{\left\{\sum_{i < j}\frac{ar_{ij}}{1+\beta r_{ij}}\right\}},
\]
which means that the gradient needed for the so-called quantum force and local energy 
can be calculated analytically.
This will speed up your code since the computation of the correlation part and the Slater determinant are the most 
time consuming parts in your code.  

We will refer to this correlation function as $\Psi_C$ or the \emph{linear Pade-Jastrow}.
\end{block}
\end{frame}

\begin{frame}[plain,fragile]
\frametitle{The first attempt at solving the Helium atom}

\begin{block}{}

We can test this by computing the local energy for our helium wave function
\[
   \psi_{T}(\bm{r}_1,\bm{r}_2) = 
   \exp{\left(-\alpha(r_1+r_2)\right)}
   \exp{\left(\frac{r_{12}}{2(1+\beta r_{12})}\right)}, 
\]
with $\alpha$ and $\beta$ as variational parameters.

The local energy is for this case 
\[ 
E_{L2} = E_{L1}+\frac{1}{2(1+\beta r_{12})^2}\left\{\frac{\alpha(r_1+r_2)}{r_{12}}(1-\frac{\bm{r}_1\bm{r}_2}{r_1r_2})-\frac{1}{2(1+\beta r_{12})^2}-\frac{2}{r_{12}}+\frac{2\beta}{1+\beta r_{12}}\right\}
\]
It is very useful to test your code against these expressions. It means also that you don't need to
compute a derivative numerically as discussed in the code example below. 
\end{block}
\end{frame}

\begin{frame}[plain,fragile]
\frametitle{The first attempt at solving the Helium atom}

\begin{block}{}
For the computation of various derivatives with different types of wave functions, you will find it useful to use python with symbolic python, that is sympy, see \href{{http://docs.sympy.org/latest/index.html}}{\nolinkurl{http://docs.sympy.org/latest/index.html}}.  Using sympy allows you autogenerate both Latex code as well c++, python or Fortran codes. Here you will find some simple examples. We choose 
the $2s$ hydrogen-orbital  (not normalized) as an example
\[
 \phi_{2s}(\bm{r}) = (Zr - 2)\exp{-(\frac{1}{2}Zr)},
\]
with $ r^2 = x^2 + y^2 + z^2$.

\begin{minted}[fontsize=\fontsize{9pt}{9pt},linenos=false,mathescape,baselinestretch=1.0,fontfamily=tt,xleftmargin=2mm]{python}
from sympy import symbols, diff, exp, sqrt
x, y, z, Z = symbols('x y z Z')
r = sqrt(x*x + y*y + z*z)
r
phi = (Z*r - 2)*exp(-Z*r/2)
phi
diff(phi, x)
\end{minted}
This doesn't look very nice, but sympy provides several functions that allow for improving and simplifying the output.
\end{block}
\end{frame}

\begin{frame}[plain,fragile]
\frametitle{The first attempt at solving the Helium atom}

\begin{block}{}
We can improve our output by factorizing and substituting expressions
\begin{minted}[fontsize=\fontsize{9pt}{9pt},linenos=false,mathescape,baselinestretch=1.0,fontfamily=tt,xleftmargin=2mm]{python}
from sympy import symbols, diff, exp, sqrt, factor, Symbol, printing
x, y, z, Z = symbols('x y z Z')
r = sqrt(x*x + y*y + z*z)
phi = (Z*r - 2)*exp(-Z*r/2)
R = Symbol('r') #Creates a symbolic equivalent of r
#print latex and c++ code
print printing.latex(diff(phi, x).factor().subs(r, R))
print printing.ccode(diff(phi, x).factor().subs(r, R))
\end{minted}
\end{block}
\end{frame}

\begin{frame}[plain,fragile]
\frametitle{The first attempt at solving the Helium atom}

\begin{block}{}
We can in turn look at second derivatives
\begin{minted}[fontsize=\fontsize{9pt}{9pt},linenos=false,mathescape,baselinestretch=1.0,fontfamily=tt,xleftmargin=2mm]{python}
from sympy import symbols, diff, exp, sqrt, factor, Symbol, printing
x, y, z, Z = symbols('x y z Z')
r = sqrt(x*x + y*y + z*z)
phi = (Z*r - 2)*exp(-Z*r/2)
R = Symbol('r') #Creates a symbolic equivalent of r
(diff(diff(phi, x), x) + diff(diff(phi, y), y) + diff(diff(phi, z), z)).factor().subs(r, R)
# Collect the Z values
(diff(diff(phi, x), x) + diff(diff(phi, y), y) +diff(diff(phi, z), z)).factor().collect(Z).subs(r, R)
# Factorize also the r**2 terms
(diff(diff(phi, x), x) + diff(diff(phi, y), y) + diff(diff(phi, z), z)).factor().collect(Z).subs(r, R).subs(r**2, R**2).factor()
print printing.ccode((diff(diff(phi, x), x) + diff(diff(phi, y), y) + diff(diff(phi, z), z)).factor().collect(Z).subs(r, R).subs(r**2, R**2).factor())
\end{minted}
With some practice this allows one to be able to check one's own calculation and translate automatically into code lines.
\end{block}
\end{frame}

\begin{frame}[plain,fragile]
\frametitle{The first attempt at solving the Helium atom}

\begin{block}{The c++ code with a VMC Solver class, main program first }

\begin{minted}[fontsize=\fontsize{9pt}{9pt},linenos=false,mathescape,baselinestretch=1.0,fontfamily=tt,xleftmargin=2mm]{c++}
#include "vmcsolver.h"
#include <iostream>
using namespace std;

int main()
{
    VMCSolver *solver = new VMCSolver();
    solver->runMonteCarloIntegration();
    return 0;
}
\end{minted}
\end{block}
\end{frame}

\begin{frame}[plain,fragile]
\frametitle{The first attempt at solving the Helium atom}

\begin{block}{The c++ code with a VMC Solver class, the VMCSolver header file }

\begin{minted}[fontsize=\fontsize{9pt}{9pt},linenos=false,mathescape,baselinestretch=1.0,fontfamily=tt,xleftmargin=2mm]{c++}
#ifndef VMCSOLVER_H
#define VMCSOLVER_H
#include <armadillo>
using namespace arma;
class VMCSolver
{
public:
    VMCSolver();
    void runMonteCarloIntegration();

private:
    double waveFunction(const mat &r);
    double localEnergy(const mat &r);
    int nDimensions;
    int charge;
    double stepLength;
    int nParticles;
    double h;
    double h2;
    long idum;
    double alpha;
    int nCycles;
    mat rOld;
    mat rNew;
};
#endif // VMCSOLVER_H
\end{minted}
\end{block}
\end{frame}

\begin{frame}[plain,fragile]
\frametitle{The first attempt at solving the Helium atom}

\begin{block}{The c++ code with a VMC Solver class, VMCSolver codes, initialize }

\begin{minted}[fontsize=\fontsize{9pt}{9pt},linenos=false,mathescape,baselinestretch=1.0,fontfamily=tt,xleftmargin=2mm]{c++}
#include "vmcsolver.h"
#include "lib.h"
#include <armadillo>
#include <iostream>
using namespace arma;
using namespace std;

VMCSolver::VMCSolver() :
    nDimensions(3),
    charge(2),
    stepLength(1.0),
    nParticles(2),
    h(0.001),
    h2(1000000),
    idum(-1),
    alpha(0.5*charge),
    nCycles(1000000)
{
}
\end{minted}
\end{block}
\end{frame}

\begin{frame}[plain,fragile]
\frametitle{The first attempt at solving the Helium atom}

\begin{block}{The c++ code with a VMC Solver class, VMCSolver codes }

\begin{minted}[fontsize=\fontsize{9pt}{9pt},linenos=false,mathescape,baselinestretch=1.0,fontfamily=tt,xleftmargin=2mm]{c++}
void VMCSolver::runMonteCarloIntegration()
{
    rOld = zeros<mat>(nParticles, nDimensions);
    rNew = zeros<mat>(nParticles, nDimensions);
    double waveFunctionOld = 0;
    double waveFunctionNew = 0;
    double energySum = 0;
    double energySquaredSum = 0;
    double deltaE;
    // initial trial positions
    for(int i = 0; i < nParticles; i++) {
        for(int j = 0; j < nDimensions; j++) {
            rOld(i,j) = stepLength * (ran2(&idum) - 0.5);
        }
    }
    rNew = rOld;
    // loop over Monte Carlo cycles
    for(int cycle = 0; cycle < nCycles; cycle++) {
        // Store the current value of the wave function
        waveFunctionOld = waveFunction(rOld);
        // New position to test
        for(int i = 0; i < nParticles; i++) {
            for(int j = 0; j < nDimensions; j++) {
                rNew(i,j) = rOld(i,j) + stepLength*(ran2(&idum) - 0.5);
            }
            // Recalculate the value of the wave function
            waveFunctionNew = waveFunction(rNew);
            // Check for step acceptance (if yes, update position, if no, reset position)
            if(ran2(&idum) <= (waveFunctionNew*waveFunctionNew) / (waveFunctionOld*waveFunctionOld)) {
                for(int j = 0; j < nDimensions; j++) {
                    rOld(i,j) = rNew(i,j);
                    waveFunctionOld = waveFunctionNew;
                }
            } else {
                for(int j = 0; j < nDimensions; j++) {
                    rNew(i,j) = rOld(i,j);
                }
            }
            // update energies
            deltaE = localEnergy(rNew);
            energySum += deltaE;
            energySquaredSum += deltaE*deltaE;
        }
    }
    double energy = energySum/(nCycles * nParticles);
    double energySquared = energySquaredSum/(nCycles * nParticles);
    cout << "Energy: " << energy << " Energy (squared sum): " << energySquared << endl;
}
\end{minted}
\end{block}
\end{frame}

\begin{frame}[plain,fragile]
\frametitle{The first attempt at solving the Helium atom}

\begin{block}{The c++ code with a VMC Solver class, VMCSolver codes }

\begin{minted}[fontsize=\fontsize{9pt}{9pt},linenos=false,mathescape,baselinestretch=1.0,fontfamily=tt,xleftmargin=2mm]{c++}
double VMCSolver::localEnergy(const mat &r)
{
    mat rPlus = zeros<mat>(nParticles, nDimensions);
    mat rMinus = zeros<mat>(nParticles, nDimensions);
    rPlus = rMinus = r;
    double waveFunctionMinus = 0;
    double waveFunctionPlus = 0;
    double waveFunctionCurrent = waveFunction(r);
    // Kinetic energy, brute force derivations
    double kineticEnergy = 0;
    for(int i = 0; i < nParticles; i++) {
        for(int j = 0; j < nDimensions; j++) {
            rPlus(i,j) += h;
            rMinus(i,j) -= h;
            waveFunctionMinus = waveFunction(rMinus);
            waveFunctionPlus = waveFunction(rPlus);
            kineticEnergy -= (waveFunctionMinus + waveFunctionPlus - 2 * waveFunctionCurrent);
            rPlus(i,j) = r(i,j);
            rMinus(i,j) = r(i,j);
        }
    }
    kineticEnergy = 0.5 * h2 * kineticEnergy / waveFunctionCurrent;
    // Potential energy
    double potentialEnergy = 0;
    double rSingleParticle = 0;
    for(int i = 0; i < nParticles; i++) {
        rSingleParticle = 0;
        for(int j = 0; j < nDimensions; j++) {
            rSingleParticle += r(i,j)*r(i,j);
        }
        potentialEnergy -= charge / sqrt(rSingleParticle);
    }
    // Contribution from electron-electron potential
    double r12 = 0;
    for(int i = 0; i < nParticles; i++) {
        for(int j = i + 1; j < nParticles; j++) {
            r12 = 0;
            for(int k = 0; k < nDimensions; k++) {
                r12 += (r(i,k) - r(j,k)) * (r(i,k) - r(j,k));
            }
            potentialEnergy += 1 / sqrt(r12);
        }
    }
    return kineticEnergy + potentialEnergy;
}
\end{minted}
\end{block}
\end{frame}

\begin{frame}[plain,fragile]
\frametitle{The first attempt at solving the Helium atom}

\begin{block}{The c++ code with a VMC Solver class, VMCSolver codes }

\begin{minted}[fontsize=\fontsize{9pt}{9pt},linenos=false,mathescape,baselinestretch=1.0,fontfamily=tt,xleftmargin=2mm]{c++}
double VMCSolver::waveFunction(const mat &r)
{
    double argument = 0;
    for(int i = 0; i < nParticles; i++) {
        double rSingleParticle = 0;
        for(int j = 0; j < nDimensions; j++) {
            rSingleParticle += r(i,j) * r(i,j);
        }
        argument += sqrt(rSingleParticle);
    }
    return exp(-argument * alpha);
}
\end{minted}
\end{block}
\end{frame}

\begin{frame}[plain,fragile]
\frametitle{The first attempt at solving the Helium atom}

\begin{block}{The c++ code with a VMC Solver class, the VMCSolver header file }

\begin{minted}[fontsize=\fontsize{9pt}{9pt},linenos=false,mathescape,baselinestretch=1.0,fontfamily=tt,xleftmargin=2mm]{c++}
#include <armadillo>
#include <iostream>
using namespace arma;
using namespace std;
double ran2(long *);

class VMCSolver
{
public:
    VMCSolver();
    void runMonteCarloIntegration();

private:
    double waveFunction(const mat &r);
    double localEnergy(const mat &r);
    int nDimensions;
    int charge;
    double stepLength;
    int nParticles;
    double h;
    double h2;
    long idum;
    double alpha;
    int nCycles;
    mat rOld;
    mat rNew;
};

VMCSolver::VMCSolver() :
    nDimensions(3),
    charge(2),
    stepLength(1.0),
    nParticles(2),
    h(0.001),
    h2(1000000),
    idum(-1),
    alpha(0.5*charge),
    nCycles(1000000)
{
}

void VMCSolver::runMonteCarloIntegration()
{
    rOld = zeros<mat>(nParticles, nDimensions);
    rNew = zeros<mat>(nParticles, nDimensions);
    double waveFunctionOld = 0;
    double waveFunctionNew = 0;
    double energySum = 0;
    double energySquaredSum = 0;
    double deltaE;
    // initial trial positions
    for(int i = 0; i < nParticles; i++) {
        for(int j = 0; j < nDimensions; j++) {
            rOld(i,j) = stepLength * (ran2(&idum) - 0.5);
        }
    }
    rNew = rOld;
    // loop over Monte Carlo cycles
    for(int cycle = 0; cycle < nCycles; cycle++) {
        // Store the current value of the wave function
        waveFunctionOld = waveFunction(rOld);
        // New position to test
        for(int i = 0; i < nParticles; i++) {
            for(int j = 0; j < nDimensions; j++) {
                rNew(i,j) = rOld(i,j) + stepLength*(ran2(&idum) - 0.5);
            }
            // Recalculate the value of the wave function
            waveFunctionNew = waveFunction(rNew);
            // Check for step acceptance (if yes, update position, if no, reset position)
            if(ran2(&idum) <= (waveFunctionNew*waveFunctionNew) / (waveFunctionOld*waveFunctionOld)) {
                for(int j = 0; j < nDimensions; j++) {
                    rOld(i,j) = rNew(i,j);
                    waveFunctionOld = waveFunctionNew;
                }
            } else {
                for(int j = 0; j < nDimensions; j++) {
                    rNew(i,j) = rOld(i,j);
                }
            }
            // update energies
            deltaE = localEnergy(rNew);
            energySum += deltaE;
            energySquaredSum += deltaE*deltaE;
        }
    }
    double energy = energySum/(nCycles * nParticles);
    double energySquared = energySquaredSum/(nCycles * nParticles);
    cout << "Energy: " << energy << " Energy (squared sum): " << energySquared << endl;
}

double VMCSolver::localEnergy(const mat &r)
{
    mat rPlus = zeros<mat>(nParticles, nDimensions);
    mat rMinus = zeros<mat>(nParticles, nDimensions);
    rPlus = rMinus = r;
    double waveFunctionMinus = 0;
    double waveFunctionPlus = 0;
    double waveFunctionCurrent = waveFunction(r);
    // Kinetic energy, brute force derivations
    double kineticEnergy = 0;
    for(int i = 0; i < nParticles; i++) {
        for(int j = 0; j < nDimensions; j++) {
            rPlus(i,j) += h;
            rMinus(i,j) -= h;
            waveFunctionMinus = waveFunction(rMinus);
            waveFunctionPlus = waveFunction(rPlus);
            kineticEnergy -= (waveFunctionMinus + waveFunctionPlus - 2 * waveFunctionCurrent);
            rPlus(i,j) = r(i,j);
            rMinus(i,j) = r(i,j);
        }
    }
    kineticEnergy = 0.5 * h2 * kineticEnergy / waveFunctionCurrent;
    // Potential energy
    double potentialEnergy = 0;
    double rSingleParticle = 0;
    for(int i = 0; i < nParticles; i++) {
        rSingleParticle = 0;
        for(int j = 0; j < nDimensions; j++) {
            rSingleParticle += r(i,j)*r(i,j);
        }
        potentialEnergy -= charge / sqrt(rSingleParticle);
    }
    // Contribution from electron-electron potential
    double r12 = 0;
    for(int i = 0; i < nParticles; i++) {
        for(int j = i + 1; j < nParticles; j++) {
            r12 = 0;
            for(int k = 0; k < nDimensions; k++) {
                r12 += (r(i,k) - r(j,k)) * (r(i,k) - r(j,k));
            }
            potentialEnergy += 1 / sqrt(r12);
        }
    }
    return kineticEnergy + potentialEnergy;
}

double VMCSolver::waveFunction(const mat &r)
{
    double argument = 0;
    for(int i = 0; i < nParticles; i++) {
        double rSingleParticle = 0;
        for(int j = 0; j < nDimensions; j++) {
            rSingleParticle += r(i,j) * r(i,j);
        }
        argument += sqrt(rSingleParticle);
    }
    return exp(-argument * alpha);
}

/*
** The function
**         ran2()
** is a long periode (> 2 x 10^18) random number generator of
** L'Ecuyer and Bays-Durham shuffle and added safeguards.
** Call with idum a negative integer to initialize; thereafter,
** do not alter idum between sucessive deviates in a
** sequence. RNMX should approximate the largest floating point value
** that is less than 1.
** The function returns a uniform deviate between 0.0 and 1.0
** (exclusive of end-point values).
*/

#define IM1 2147483563
#define IM2 2147483399
#define AM (1.0/IM1)
#define IMM1 (IM1-1)
#define IA1 40014
#define IA2 40692
#define IQ1 53668
#define IQ2 52774
#define IR1 12211
#define IR2 3791
#define NTAB 32
#define NDIV (1+IMM1/NTAB)
#define EPS 1.2e-7
#define RNMX (1.0-EPS)

double ran2(long *idum)
{
  int            j;
  long           k;
  static long    idum2 = 123456789;
  static long    iy=0;
  static long    iv[NTAB];
  double         temp;

  if(*idum <= 0) {
    if(-(*idum) < 1) *idum = 1;
    else             *idum = -(*idum);
    idum2 = (*idum);
    for(j = NTAB + 7; j >= 0; j--) {
      k     = (*idum)/IQ1;
      *idum = IA1*(*idum - k*IQ1) - k*IR1;
      if(*idum < 0) *idum +=  IM1;
      if(j < NTAB)  iv[j]  = *idum;
    }
    iy=iv[0];
  }
  k     = (*idum)/IQ1;
  *idum = IA1*(*idum - k*IQ1) - k*IR1;
  if(*idum < 0) *idum += IM1;
  k     = idum2/IQ2;
  idum2 = IA2*(idum2 - k*IQ2) - k*IR2;
  if(idum2 < 0) idum2 += IM2;
  j     = iy/NDIV;
  iy    = iv[j] - idum2;
  iv[j] = *idum;
  if(iy < 1) iy += IMM1;
  if((temp = AM*iy) > RNMX) return RNMX;
  else return temp;
}
#undef IM1
#undef IM2
#undef AM
#undef IMM1
#undef IA1
#undef IA2
#undef IQ1
#undef IQ2
#undef IR1
#undef IR2
#undef NTAB
#undef NDIV
#undef EPS
#undef RNMX

// End: function ran2()


#include <iostream>
using namespace std;

int main()
{
    VMCSolver *solver = new VMCSolver();
    solver->runMonteCarloIntegration();
    return 0;
}
\end{minted}
\end{block}
\end{frame}

\begin{frame}[plain,fragile]
\frametitle{The first attempt at solving the Helium atom}

\begin{block}{Exercises for first lab session, Thursday 22 }

\begin{itemize}
 \item If you have never used git, Qt, armadillo etc, get familiar with them, see the guides at the official UiO website of the course.

 \item Study the simple program in the folder \href{{https://github.com/CompPhysics/ComputationalPhysics2/tree/gh-pages/doc/pub/vmc/programs/}}{\nolinkurl{https://github.com/CompPhysics/ComputationalPhysics2/tree/gh-pages/doc/pub/vmc/programs/}}

 \item Implement the closed form expression for the local energy and the so-called quantum force

 \item Convince yourself that the closed form expressions are correct. Check both wave functions

 \item Implement the closed form expression for the local energy and compare with a code where the second derivatives are computed numerically.
\end{itemize}

\noindent
\end{block}
\end{frame}

\begin{frame}[plain,fragile]
\frametitle{The Metropolis algorithm}

\begin{block}{}
The Metropolis algorithm , see \href{{http://scitation.aip.org/content/aip/journal/jcp/21/6/10.1063/1.1699114}}{\nolinkurl{http://scitation.aip.org/content/aip/journal/jcp/21/6/10.1063/1.1699114}}  (see also the FYS3150 lectures \href{{http://www.uio.no/studier/emner/matnat/fys/FYS3150/h14/index.html}}{\nolinkurl{http://www.uio.no/studier/emner/matnat/fys/FYS3150/h14/index.html}})
was invented by Metropolis et. al
and is often simply called the Metropolis algorithm.
It is a method to sample a normalized probability
distribution by a stochastic process. We define ${\cal P}_i^{(n)}$ to
be the probability for finding the system in the state $i$ at step $n$.
The algorithm is then

\begin{itemize}
\item Sample a possible new state $j$ with some probability $T_{i\rightarrow j}$.

\item Accept the new state $j$ with probability $A_{i \rightarrow j}$ and use it as the next sample. With probability $1-A_{i\rightarrow j}$ the move is rejected and the original state $i$ is used again as a sample.
\end{itemize}

\noindent
\end{block}
\end{frame}

\begin{frame}[plain,fragile]
\frametitle{The Metropolis algorithm}

\begin{block}{}
We wish to derive the required properties of $T$ and $A$ such that
${\cal P}_i^{(n\rightarrow \infty)} \rightarrow p_i$ so that starting
from any distribution, the method converges to the correct distribution.
Note that the description here is for a discrete probability distribution.
Replacing probabilities $p_i$ with expressions like $p(x_i)dx_i$ will
take all of these over to the corresponding continuum expressions.

\end{block}
\end{frame}

\begin{frame}[plain,fragile]
\frametitle{The Metropolis algorithm}

\begin{block}{}
The dynamical equation for ${\cal P}_i^{(n)}$ can be written directly from
the description above. The probability of being in the state $i$ at step $n$
is given by the probability of being in any state $j$ at the previous step,
and making an accepted transition to $i$ added to the probability of
being in the state $i$, making a transition to any state $j$ and
rejecting the move:
\[
{\cal P}^{(n)}_i = \sum_j \left [
{\cal P}^{(n-1)}_jT_{j\rightarrow i} A_{j\rightarrow i} 
+{\cal P}^{(n-1)}_iT_{i\rightarrow j}\left ( 1- A_{i\rightarrow j} \right)
\right ] \,.
\]
Since the probability of making some transition must be 1,
$\sum_j T_{i\rightarrow j} = 1$, and the above equation becomes
\[
{\cal P}^{(n)}_i = {\cal P}^{(n-1)}_i +
 \sum_j \left [
{\cal P}^{(n-1)}_jT_{j\rightarrow i} A_{j\rightarrow i} 
-{\cal P}^{(n-1)}_iT_{i\rightarrow j}A_{i\rightarrow j}
\right ] \,.
\]
\end{block}
\end{frame}

\begin{frame}[plain,fragile]
\frametitle{The Metropolis algorithm}

\begin{block}{}
For large $n$ we require that ${\cal P}^{(n\rightarrow \infty)}_i = p_i$,
the desired probability distribution. Taking this limit, gives the
balance requirement
\[
 \sum_j \left [
p_jT_{j\rightarrow i} A_{j\rightarrow i}
-p_iT_{i\rightarrow j}A_{i\rightarrow j}
\right ] = 0 \,.
\]
The balance requirement is very weak. Typically the much stronger detailed
balance requirement is enforced, that is rather than the sum being
set to zero, we set each term separately to zero and use this
to determine the acceptance probabilities. Rearranging, the result is
\[
\frac{ A_{j\rightarrow i}}{A_{i\rightarrow j}}
= \frac{p_iT_{i\rightarrow j}}{ p_jT_{j\rightarrow i}} \,.
\]
\end{block}
\end{frame}

\begin{frame}[plain,fragile]
\frametitle{The Metropolis algorithm}

\begin{block}{}
The Metropolis choice is to maximize the $A$ values, that is
\[
A_{j \rightarrow i} = \min \left ( 1,
\frac{p_iT_{i\rightarrow j}}{ p_jT_{j\rightarrow i}}\right ).
\]
Other choices are possible, but they all correspond to multilplying
$A_{i\rightarrow j}$ and $A_{j\rightarrow i}$ by the same constant
smaller than unity.\footnote{The penalty function method uses just such
a factor to compensate for $p_i$ that are evaluated stochastically
and are therefore noisy.}
\end{block}
\end{frame}

\begin{frame}[plain,fragile]
\frametitle{The Metropolis algorithm}

\begin{block}{}
Having chosen the acceptance probabilities, we have guaranteed that
if the  ${\cal P}_i^{(n)}$ has equilibrated, that is if it is equal to $p_i$,
it will remain equilibrated. Next we need to find the circumstances for
convergence to equilibrium.

The dynamical equation can be written as
\[
{\cal P}^{(n)}_i = \sum_j M_{ij}{\cal P}^{(n-1)}_j
\]
with the matrix $M$ given by
\[
M_{ij} = \delta_{ij}\left [ 1 -\sum_k T_{i\rightarrow k} A_{i \rightarrow k}
\right ] + T_{j\rightarrow i} A_{j\rightarrow i} \,.
\]
Summing over $i$ shows that $\sum_i M_{ij} = 1$, and since
$\sum_k T_{i\rightarrow k} = 1$, and $A_{i \rightarrow k} \leq 1$, the
elements of the matrix satisfy $M_{ij} \geq 0$. The matrix $M$ is therefore
a stochastic matrix.

\end{block}
\end{frame}

\begin{frame}[plain,fragile]
\frametitle{The Metropolis algorithm}

\begin{block}{}
The Metropolis method is simply the power method for computing the
right eigenvector of $M$ with the largest magnitude eigenvalue.
By construction, the correct probability distribution is a right eigenvector
with eigenvalue 1. Therefore, for the Metropolis method to converge
to this result, we must show that $M$ has only one eigenvalue with this
magnitude, and all other eigenvalues are smaller.

\end{block}
\end{frame}

\end{document}
