%%
%% Automatically generated file from DocOnce source
%% (https://github.com/hplgit/doconce/)
%%


%-------------------- begin preamble ----------------------

\documentclass[%
twoside,                 % oneside: electronic viewing, twoside: printing
final,                   % or draft (marks overfull hboxes, figures with paths)
10pt]{article}

\listfiles               % print all files needed to compile this document

\usepackage{relsize,makeidx,color,setspace,amsmath,amsfonts}
\usepackage[table]{xcolor}
\usepackage{bm,microtype}

\usepackage{fancyvrb} % packages needed for verbatim environments

\usepackage[T1]{fontenc}
%\usepackage[latin1]{inputenc}
\usepackage[utf8]{inputenc}

\usepackage{lmodern}         % Latin Modern fonts derived from Computer Modern

% Hyperlinks in PDF:
\definecolor{linkcolor}{rgb}{0,0,0.4}
\usepackage[%
    colorlinks=true,
    linkcolor=linkcolor,
    urlcolor=linkcolor,
    citecolor=black,
    filecolor=black,
    %filecolor=blue,
    pdfmenubar=true,
    pdftoolbar=true,
    bookmarksdepth=3   % Uncomment (and tweak) for PDF bookmarks with more levels than the TOC
            ]{hyperref}
%\hyperbaseurl{}   % hyperlinks are relative to this root

\setcounter{tocdepth}{2}  % number chapter, section, subsection

\usepackage[framemethod=TikZ]{mdframed}

% --- begin definitions of admonition environments ---

% --- end of definitions of admonition environments ---

% prevent orhpans and widows
\clubpenalty = 10000
\widowpenalty = 10000

% --- end of standard preamble for documents ---


% insert custom LaTeX commands...

\raggedbottom
\makeindex

%-------------------- end preamble ----------------------

\begin{document}



% ------------------- main content ----------------------



% ----------------- title -------------------------

\thispagestyle{empty}

\begin{center}
{\LARGE\bf
\begin{spacing}{1.25}
Slides from FYS4411/9411 Computational Physics II Lectures
\end{spacing}
}
\end{center}

% ----------------- author(s) -------------------------

\begin{center}
{\bf Morten Hjorth-Jensen${}^{1, 2}$} \\ [0mm]
\end{center}

    \begin{center}
% List of all institutions:
\centerline{{\small ${}^1$Department of Physics, University of Oslo, Oslo, Norway}}
\centerline{{\small ${}^2$Department of Physics and Astronomy and National Superconducting Cyclotron Laboratory, Michigan State University, East Lansing, Michigan, USA}}
\end{center}
    
% ----------------- end author(s) -------------------------

\begin{center} % date
Spring 2015
\end{center}

\vspace{1cm}


% !split
\subsection*{Aims}

% --- begin paragraph admon ---
\paragraph{}
\begin{itemize}
\item Be able to apply two central methods, Variational Monte Carlo and Hartree-Fock (and density functional theory) to properties of atoms and molecules.

\item Understand how to simulate qauntum mechanical systems with many interacting particles. The methods are relevant for atomic, molecular, solid state, materials science, nanotechnology, quantum chemistry  and nuclear physics. 
\end{itemize}

\noindent
% --- end paragraph admon ---





% !split
\subsection*{Overview of first week}


% --- begin paragraph admon ---
\paragraph{}
\begin{itemize}
  \item Monday 9.15am-12pm: First lecture: Presentation of the course, aims and content

  \item Monday 9.15am-12pm: Introduction to Monte Carlo methods and basic many-body physics

  \item Tuesday 9.15am-12pm: Basic many-body physics and variational Monte Carlo methods

  \item Wednesday 9.15am-12pm: Variational Monte Carlo methods and presentation of project 1

  \item Thursday 2.15pm-4pm: Continued discussion of first project, variational Monte Carlo methods and codes

  \item Computer lab: Thursday 4pm-7pm. First time: Thursday this week at room FV329.
\end{itemize}

\noindent
% --- end paragraph admon ---



% !split
\subsection*{Lectures and ComputerLab}


% --- begin paragraph admon ---
\paragraph{}
\begin{itemize}
  \item Lectures: Thursday (2.15pm-4pm), remotely via adove connect every second week.

  \item Computerlab: Thursday (2.15pm-7pm), first time January 22, last lab session May 14.

  \item Weekly plans and all other information are on the webpage, see URL: ''

  \item Second intensive week starts April 7 and ends April 10.

  \item First project to be handed in February 27 at noon.

  \item Second project to be handed in March 27 at noon.

  \item Third and final project to be handed in June 1 at noon. 
\end{itemize}

\noindent
% --- end paragraph admon ---



% !split
\subsection*{Course Format}


% --- begin paragraph admon ---
\paragraph{}
\begin{itemize}
  \item Three  compulsory projects. Electronic reports only. You are free to choose your format.

  \item Evaluation and grading: The first two projects count 25\% while the last project counts 50\% of the final grade. There is no oral or written exam.

  \item The computer lab (room FV329) consists of 16 Linux PCs, but many prefer own laptops. C/C++ is the default programming language, but Fortran2008 and Python are also used. All source codes discussed during the lectures can be found at the webpage of the course. We recommend either C/C++, Fortran2008 or Python as languages.
\end{itemize}

\noindent
% --- end paragraph admon ---




% !split
\subsection*{Topics covered in this course}

% --- begin paragraph admon ---
\paragraph{}
\begin{itemize}
  \item Parallelization (MPI and OpenMP), high-performance computing topics. Choose between Fortran2008 and/or C++ as programming languages. Python also possible as programming language. 

  \item Algorithms for Monte Carlo Simulations (multidimensional integrals), Metropolis-Hastings and importance sampling algorithms.  Improved Monte Carlo methods.

  \item Statistical analysis of data  from Monte Carlo calculations, blocking method. (exercise 1 of part 1)

  \item Eigenvalue solvers, efficient computations of integrals
\end{itemize}

\noindent
% --- end paragraph admon ---



% !split
\subsection*{Topics covered in this course}

% --- begin paragraph admon ---
\paragraph{}
\begin{itemize}
  \item Search for minima in multidimensional spaces (conjugate gradient method, steepest descent method, quasi-Newton-Raphson, Broyden-Jacobian).

  \item Iterative methods for solutions of non-linear equations.

  \item Object orientation

  \item Variational Monte Carlo for 'ab initio' studies of quantum mechanical many-body systems.

  \item Simulation of three-dimensional systems like atoms or molecules.

  \item Hartree-Fock method to study atoms and molecules
\end{itemize}

\noindent
% --- end paragraph admon ---





% !split
\subsection*{Quantum Monte Carlo Motivation}

% --- begin paragraph admon ---
\paragraph{}
Most quantum mechanical  problems of interest in for example atomic, molecular, nuclear and solid state 
physics consist of a large number of interacting electrons and ions or nucleons. 

The total number of particles $N$ is usually sufficiently large
that an exact solution cannot be found. 

Typically, 
the expectation value for a chosen hamiltonian for a system of  $N$ particles is
\[
   \langle H \rangle =
   \frac{\int d\bm{R}_1d\bm{R}_2\dots d\bm{R}_N
         \Psi^{\ast}(\bm{R_1},\bm{R}_2,\dots,\bm{R}_N)
          H(\bm{R_1},\bm{R}_2,\dots,\bm{R}_N)
          \Psi(\bm{R_1},\bm{R}_2,\dots,\bm{R}_N)}
        {\int d\bm{R}_1d\bm{R}_2\dots d\bm{R}_N
        \Psi^{\ast}(\bm{R_1},\bm{R}_2,\dots,\bm{R}_N)
        \Psi(\bm{R_1},\bm{R}_2,\dots,\bm{R}_N)},
\]
an in general intractable problem.

 This integral is actually the starting point in a Variational Monte Carlo calculation. \textbf{Gaussian quadrature: Forget it}! Given 10 particles and 10 mesh points for each degree of freedom
and an
 ideal 1 Tflops machine (all operations take the same time), how long will it take to compute the above integral? The lifetime of the universe is of the order of $10^{17}$ s.
% --- end paragraph admon ---




% !split
\subsection*{Quantum Monte Carlo Motivation}

% --- begin paragraph admon ---
\paragraph{}
As an example from the nuclear many-body problem, we have Schroedinger's equation as a differential equation
\[
  \hat{H}\Psi(\bm{r}_1,..,\bm{r}_A,\alpha_1,..,\alpha_A)=E\Psi(\bm{r}_1,..,\bm{r}_A,\alpha_1,..,\alpha_A)
\]
where
\[
  \bm{r}_1,..,\bm{r}_A,
\]
are the coordinates and 
\[
  \alpha_1,..,\alpha_A,
\]
are sets of relevant quantum numbers such as spin and isospin for a system of  $A$ nucleons ($A=N+Z$, $N$ being the number of neutrons and $Z$ the number of protons).
% --- end paragraph admon ---




% !split
\subsection*{Quantum Monte Carlo Motivation}

% --- begin paragraph admon ---
\paragraph{}
There are
\[
 2^A\times \left(\begin{array}{c} A\\ Z\end{array}\right)
\]
coupled second-order differential equations in $3A$ dimensions.

For a nucleus like ${}^{10}\mbox{Be}$ this number is \textbf{215040}.
This is a truely challenging many-body problem.

Methods like partial differential equations can at most be used for 2-3 particles.
% --- end paragraph admon ---




% !split
\subsection*{Quantum Monte Carlo Motivation}

% --- begin paragraph admon ---
\paragraph{}
\begin{itemize}
\item Monte-Carlo methods

\item Renormalization group (RG) methods, in particular density matrix RG

\item Large-scale diagonalization (Iterative methods, Lanczo's method, dimensionalities  $10^{10}$ states)

\item Coupled cluster theory, favoured method in quantum chemistry, molecular and atomic physics. Applications to ab initio calculations in nuclear physics as well for large nuclei.

\item Perturbative many-body methods 

\item Green's function methods

\item Density functional theory/Mean-field theory and Hartree-Fock theory
\end{itemize}

\noindent
The physics of the system hints at which many-body methods to use.
% --- end paragraph admon ---





% !split
\subsection*{Quantum Monte Carlo Motivation}

% --- begin paragraph admon ---
\paragraph{Pros and Cons of Monte Carlo.}
\begin{itemize}
\item Is physically intuitive.

\item Allows one to study systems with many degrees of freedom. Diffusion Monte Carlo (DMC) and Green's function Monte Carlo (GFMC) yield in principle the exact solution to Schroedinger's equation.

\item Variational Monte Carlo (VMC) is easy  to implement but needs a reliable trial wave function, can be difficult to obtain.  This is where we will use Hartree-Fock theory to construct an optimal basis.

\item DMC/GFMC for fermions (spin with half-integer values, electrons, baryons, neutrinos, quarks)  has a sign problem. Nature prefers an anti-symmetric wave function. PDF in this case given distribution of random walkers ($p\ge 0$).

\item The solution has a statistical error, which can be large. 

\item There is a limit for how large systems one can study, DMC needs a huge number of random walkers in order to achieve stable results. 

\item Obtain only the lowest-lying states with a given symmetry. Can get excited states.
\end{itemize}

\noindent
% --- end paragraph admon ---





% !split
\subsection*{Quantum Monte Carlo Motivation}

% --- begin paragraph admon ---
\paragraph{Where and why do we use Monte Carlo Methods in Quantum Physics.}
\begin{itemize}
\item Quantum systems with many particles at finite temperature: Path Integral Monte Carlo with applications to dense matter and quantum liquids (phase transitions from normal fluid to superfluid). Strong correlations.

\item Bose-Einstein condensation of dilute gases, method transition from  non-linear PDE to Diffusion Monte Carlo as density increases.

\item Light atoms, molecules, solids and nuclei. 

\item Lattice Quantum-Chromo Dynamics. Impossible to solve without MC calculations. 

\item Simulations of systems in solid state physics, from semiconductors to spin systems. Many electrons active and possibly strong correlations.
\end{itemize}

\noindent
% --- end paragraph admon ---



% !split
\subsection*{Quantum Monte Carlo Motivation}

% --- begin paragraph admon ---
\paragraph{}
Given a hamiltonian $H$ and a trial wave function $\Psi_T$, the variational principle states that the expectation value of $\langle H \rangle$, defined through 
\[
   E[H]= \langle H \rangle =
   \frac{\int d\bm{R}\Psi^{\ast}_T(\bm{R})H(\bm{R})\Psi_T(\bm{R})}
        {\int d\bm{R}\Psi^{\ast}_T(\bm{R})\Psi_T(\bm{R})},
\]
is an upper bound to the ground state energy $E_0$ of the hamiltonian $H$, that is 
\[
    E_0 \le \langle H \rangle .
\]
In general, the integrals involved in the calculation of various  expectation values  are multi-dimensional ones. Traditional integration methods such as the Gauss-Legendre will not be adequate for say the  computation of the energy of a many-body system.
% --- end paragraph admon ---



% !split
\subsection*{Quantum Monte Carlo Motivation}

% --- begin paragraph admon ---
\paragraph{}
The trial wave function can be expanded in the eigenstates of the hamiltonian since they form a complete set, viz.,
\[
   \Psi_T(\bm{R})=\sum_i a_i\Psi_i(\bm{R}),
\]
and assuming the set of eigenfunctions to be normalized one obtains 
\[
     \frac{\sum_{nm}a^*_ma_n \int d\bm{R}\Psi^{\ast}_m(\bm{R})H(\bm{R})\Psi_n(\bm{R})}
        {\sum_{nm}a^*_ma_n \int d\bm{R}\Psi^{\ast}_m(\bm{R})\Psi_n(\bm{R})} =\frac{\sum_{n}a^2_n E_n}
        {\sum_{n}a^2_n} \ge E_0,
\]
where we used that $H(\bm{R})\Psi_n(\bm{R})=E_n\Psi_n(\bm{R})$.
In general, the integrals involved in the calculation of various  expectation
values  are multi-dimensional ones. 
The variational principle yields the lowest state of a given symmetry.
% --- end paragraph admon ---




% !split
\subsection*{Quantum Monte Carlo Motivation}

% --- begin paragraph admon ---
\paragraph{}
In most cases, a wave function has only small values in large parts of 
configuration space, and a straightforward procedure which uses
homogenously distributed random points in configuration space 
will most likely lead to poor results. This may suggest that some kind
of importance sampling combined with e.g., the Metropolis algorithm 
may be  a more efficient way of obtaining the ground state energy.
The hope is then that those regions of configurations space where
the wave function assumes appreciable values are sampled more 
efficiently.
% --- end paragraph admon ---




% !split
\subsection*{Quantum Monte Carlo Motivation}

% --- begin paragraph admon ---
\paragraph{}
The tedious part in a VMC calculation is the search for the variational
minimum. A good knowledge of the system is required in order to carry out
reasonable VMC calculations. This is not always the case, 
and often VMC calculations 
serve rather as the starting
point for so-called diffusion Monte Carlo calculations (DMC). DMC is a way of
solving exactly the many-body Schroedinger equation by means of 
a stochastic procedure. A good guess on the binding energy
and its wave function is however necessary. 
A carefully performed VMC calculation can aid in this context.
% --- end paragraph admon ---




% !split
\subsection*{Quantum Monte Carlo Motivation}

% --- begin paragraph admon ---
\paragraph{}
\begin{itemize}
\item Construct first a trial wave function $\psi_T(\bm{R},\bm{\alpha})$,  for a many-body system consisting of $N$ particles located at positions 
\end{itemize}

\noindent
$\bm{R}=(\bm{R}_1,\dots ,\bm{R}_N)$. The trial wave function depends on $\alpha$ variational parameters $\bm{\alpha}=(\alpha_1,\dots ,\alpha_M)$.
\begin{itemize}
\item Then we evaluate the expectation value of the hamiltonian $H$ 
\end{itemize}

\noindent
\[
   E[H]=\langle H \rangle =
   \frac{\int d\bm{R}\Psi^{\ast}_{T}(\bm{R},\bm{\alpha})H(\bm{R})\Psi_{T}(\bm{R},\bm{\alpha})}
        {\int d\bm{R}\Psi^{\ast}_{T}(\bm{R},\bm{\alpha})\Psi_{T}(\bm{R},\bm{\alpha})}.
\]
\begin{itemize}
\item Thereafter we vary $\alpha$ according to some minimization algorithm and return to the first step.
\end{itemize}

\noindent
% --- end paragraph admon ---




% !split
\subsection*{Quantum Monte Carlo Motivation}

% --- begin paragraph admon ---
\paragraph{Basic steps.}
Choose a trial wave function
$\psi_T(\bm{R})$.
\[
   P(\bm{R})= \frac{\left|\psi_T(\bm{R})\right|^2}{\int \left|\psi_T(\bm{R})\right|^2d\bm{R}}.
\]
This is our new probability distribution function  (PDF).
The approximation to the expectation value of the Hamiltonian is now 
\[
   E[H(\bm{\alpha})] = 
   \frac{\int d\bm{R}\Psi^{\ast}_T(\bm{R},\bm{\alpha})H(\bm{R})\Psi_T(\bm{R},\bm{\alpha})}
        {\int d\bm{R}\Psi^{\ast}_T(\bm{R},\bm{\alpha})\Psi_T(\bm{R},\bm{\alpha})}.
\]
% --- end paragraph admon ---




% !split
\subsection*{Quantum Monte Carlo Motivation}

% --- begin paragraph admon ---
\paragraph{}
Define a new quantity
\[
   E_L(\bm{R},\bm{\alpha})=\frac{1}{\psi_T(\bm{R},\bm{\alpha})}H\psi_T(\bm{R},\bm{\alpha}),
   \label{eq:locale1}
\]
called the local energy, which, together with our trial PDF yields
\[
  E[H(\bm{\alpha})]= = \int P(\bm{R})E_L(\bm{R}) d\bm{R}\approx \frac{1}{N}\sum_{i=1}^NP(\bm{R_i},\bm{\alpha})E_L(\bm{R_i},\bm{\alpha})
  \label{eq:vmc1}
\]
with $N$ being the number of Monte Carlo samples.
% --- end paragraph admon ---




% ------------------- end of main content ---------------


\printindex

\end{document}

